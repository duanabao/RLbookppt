\usepackage{graphicx}
\usepackage{amsmath}
\usepackage{amssymb}
\usepackage[font=footnotesize]{caption} % set the captain font size to 8 (i.e. footnotesize)
\usepackage{subfig} % uses subfloats within a single float MUST after the package {caption}!!
\usepackage{natbib}
%\usepackage{cite} % sort the reference in the article by number or alphabatic
\usepackage{color}
\usepackage{algorithm} % options: boxed [section]
\usepackage{algpseudocode} % for algorithm
%\usepackage{enumerate}
\usepackage{enumitem} % directly use itemize, easily specify indent and everything
\setlist[itemize]{leftmargin=*,label=$\bullet$}%leftmargin=*,itemsep=0pt} %topsep=5pt
\setlist[enumerate]{label={\arabic*)}}
\usepackage{hyperref}
\usepackage{wrapfig}
\usepackage{textpos}
\usepackage{bibentry} % for publication list
\makeatletter\let\saved@bibitem\@bibitem\makeatother % make hyperref and bibentry compatible!!!
\nobibliography*
\usepackage{fancybox}% shadow for image
%\usepackage{empheq} % emphasize equations
\usepackage{bm}
\usepackage{arydshln} % for dashline in table or matrix
\linespread{1.3}
\usepackage{multimedia}

\usepackage{setspace} \setstretch{1.2}

\usepackage{framed}
\colorlet{shadecolor}{black!5}
% for box, page breakable, very good!!
\usepackage[framemethod=TikZ]{mdframed}%
\mdfdefinestyle{myFrame}{%
    linecolor=gray!15!white,%gray
    outerlinewidth=0.1pt,
    roundcorner=3pt,
    skipabove=15pt, % the space before the entire box
    skipbelow=15pt, % the space after the entire box. Please see the figure 2 in the manual, very clear!
    innertopmargin=10pt,%\baselineskip,
    innerbottommargin=10pt,%\baselineskip,
    %innerrightmargin=10pt,
    %innerleftmargin=10pt,
    splittopskip=\baselineskip,
    splitbottomskip=\baselineskip,
    backgroundcolor=gray!10!white,
    frametitlerule=true,
    frametitlebackgroundcolor=gray!20!white,
    frametitleaboveskip=5pt,
    frametitlebelowskip=5pt,
}
\mdfdefinestyle{myAlgo}{%
    linecolor=gray!100!white,%gray
    outerlinewidth=0.1pt,
    roundcorner=3pt,
    skipabove=15pt, % the space before the entire box
    skipbelow=15pt, % the space after the entire box. Please see the figure 2 in the manual, very clear!
    innertopmargin=10pt,%\baselineskip,
    innerbottommargin=10pt,%\baselineskip,
    %innerrightmargin=10pt,
    %innerleftmargin=10pt,
    splittopskip=\baselineskip,
    splitbottomskip=\baselineskip,
    backgroundcolor=gray!0!white,
    frametitlerule=true,
    frametitlebackgroundcolor=gray!20!white,
    frametitleaboveskip=5pt,
    frametitlebelowskip=5pt,
}


\usepackage{tikz}
\usetikzlibrary{calc} % for calculation functions in Tikz let, in commands in Tikz
\usetikzlibrary{shapes} % for block diagram
\usetikzlibrary{chains}
\usetikzlibrary{fit}
\usetikzlibrary{arrows}
\usetikzlibrary{decorations.text} % text along path

\newcommand{\blue}[1]{\textcolor{blue}{#1}}
\definecolor{myred}{RGB}{200,0,0}
\newcommand{\red}[1]{\textcolor{myred}{#1}} %magenta purple
\newcommand{\I}{\mathcal{I}}
\newcommand{\tr}{\mathrm{tr}}
\newcommand{\Null}{\mathrm{Null}}
\newcommand{\Range}{\mathrm{Range}}
\newcommand{\one}{\mathbf{1}}
\newcommand{\rank}{\mathrm{rank}}
\newcommand{\myspan}{\mathrm{span}}
\newcommand{\mydiag}{\mathrm{diag}}
\newcommand{\D}{\mathrm{d}}
\renewcommand{\d}{\mathrm{d}}
\newcommand{\blkdiag}{\mathrm{blkdiag}}
\newcommand{\sgn}{\mathrm{sgn}}
\newcommand{\T}{\mathrm{T}}
\newcommand{\myqed}{\hfill$\blacksquare$}
\newcommand{\ep}{\varepsilon}
\newcommand{\sig}{\mathrm{sig}_a}
%\newcommand{\sigep_}[1]{\sig(\ep_{#1})}
\newcommand{\R}{\mathbb{R}}
\newcommand{\A}{\mathcal{A}}
\newcommand{\G}{\mathcal{G}}
\newcommand{\E}{\mathbb{E}}
\newcommand{\X}{\mathcal{X}}
\newcommand{\V}{\mathcal{V}}
\newcommand{\N}{\mathcal{N}}
\newcommand{\M}{\mathcal{M}}
\renewcommand{\H}{\mathcal{H}}
\renewcommand{\L}{\mathcal{B}}
\renewcommand{\S}{\mathcal{S}}
\newcommand{\xe}{x_{\text{e}}}
%\newcommand{\Null}[1]{\mathrm{Null}\left(#1\right)}
\newcommand{\sk}[1]{\left[#1\right]_\times} % skew symmetric operator
\newcommand{\dia}[1]{\mathrm{diag}\left(#1\right)} % block diagnal matrix
%\renewcommand{\span}[1]{\mathrm{span}\left\{#1\right\}} % ERROR when redefine \span
\newcommand{\Var}{\mathrm{Var}}
\newcommand{\var}{\mathrm{var}}


\graphicspath{{figures/}}

% for tikz, theorem, lemma ... environments have already been declared. You don't need to declare, or you need to use other names than theorem or lemma, such as my_theorem.
%\newtheorem{my_theorem}{Theorem}
%\newtheorem{my_lemma}{Lemma}
\newtheorem{assumption}{Assumption} % necessary for beamer
%\newtheorem{my_remark}{Remark}
\newtheorem{proposition}{Proposition} % necessary for beamer
%\newtheorem{my_corollary}{Corollary}
%\newtheorem{my_example}{Example}
%\newtheorem{my_definition}{Definition}
%\newtheorem{my_problem}{Problem}

%##################################################
\newcommand{\pagetitle}[1]{\textbf{\textcolor{BlueViolet}{$\circ$ #1}}} %!!!
\newcommand{\pagehighlight}[1]{\textbf{\textcolor{Brown}{#1}}} %!!!
\newcommand{\mypause}{\pause} % this is useful for slide show. if you don't want pause any more, just set it as blank
%\newcommand{\mybullet}{\textcolor{BlueViolet}{$\blacksquare$} }%{$\rhd$ }
%\newcommand{\myhighsign}{$\star$ }% the sign to highlight a sentence

%##################################################
% To highlight equation. Example: \begin{align*} \boxed{xxx} \end{align*}
% does not support multiline equations
% put color to \boxed math command
\newcommand*{\boxcolor}{gray}
\makeatletter
\renewcommand{\boxed}[1]{\textcolor{\boxcolor}{%
%\tikz[baseline={([yshift=-1ex]current bounding box.center)}] \node [rectangle, minimum width=1ex,rounded corners,draw] {\normalcolor\m@th$\displaystyle#1$};}}
\tikz[baseline={([yshift=-1ex]current bounding box.center)}] \node [rectangle, minimum width=2ex,rounded corners,draw] {\normalcolor\m@th$\displaystyle#1$};}}
\makeatother

%##################################################
% set my own theme
\def\structureHeight{9mm}
\usetheme[height=\structureHeight]{Rochester}
\usecolortheme[RGB={0,0,128}]{structure}
\setbeamertemplate{items}[circle]%rectangle, triangle,circle
\setbeamertemplate{blocks}[rounded][shadow=true]
\setbeamertemplate{navigation symbols}{}
%\addtobeamertemplate{frametitle}{} % specify the logo
%{
%    \begin{textblock*}{100mm}(.87\textwidth,-\structureHeight)
%        \includegraphics[height=6.6mm,width=3cm,keepaspectratio]{../common_figures_private/westlake_logo.png} % add logo
%    \end{textblock*}
%}
\addtobeamertemplate{frametitle}{\vskip4pt}{} % specify
%\setbeamerfont{frametitle}{size=\large}
\definecolor{mylightgray}{RGB}{240 240 240}
\definecolor{mykhaki}{RGB}{240 230 140}% khaki color
\definecolor{mylightYellow}{RGB}{255,255,224} % light yellow
%\setbeamercolor{beamercolor1}{bg=mylightgray, fg=black}
%\setbeamercolor{beamercolor2}{bg=mylightYellow,fg=black}%{bg=yellow!90!white, fg=black}
% background and foreground color
\setbeamercolor{background canvas}{bg=black!0!white} % background color of every slide! My previous value was black!10!white for my lecture videos!
\setbeamercolor{normal text}{bg=black!10!white} % background color for e.g. theorem environment. When canvas is 10, here it can be 20; the bg for normal text changes the color of hidden text when you use overlay

\setbeamercolor{block title}{bg=mykhaki,fg=black}
\defbeamertemplate{footline}{zsy_frameNumber}
{%
  \hspace{5pt}  \emph{Shiyu Zhao}
  \hspace*{\fill}%
  \usebeamercolor[fg]{page number in head/foot}%
  \insertframenumber\,/\,\inserttotalframenumber \vspace{0pt} \hspace{5pt}
  \vskip5pt
}
\setbeamertemplate{footline}[zsy_frameNumber]
%##################################################
\setbeamercovered{transparent=0} % a good value for transparent text is 20
% when using the overlay commands like \onslide or \uncover, the text will NOT be invisible, instead it will be like transparent
%\pause will also have the transparent effect: command \pause is easy to use: to make it invisible, change the value to zero.

