\documentclass[xcolor=dvipsnames,mathserif,9pt]{beamer} %handout
%\usefonttheme{serif}%{structurebold}%{structuresmallcapsserif}%{serif}

%\input{before_document}
\usepackage{graphicx}
\usepackage{amsmath}
\usepackage{amssymb}
\usepackage[font=footnotesize]{caption} % set the captain font size to 8 (i.e. footnotesize)
\usepackage{subfig} % uses subfloats within a single float MUST after the package {caption}!!
\usepackage{natbib}
%\usepackage{cite} % sort the reference in the article by number or alphabatic
\usepackage{color}
\usepackage{algorithm} % options: boxed [section]
\usepackage{algpseudocode} % for algorithm
%\usepackage{enumerate}
\usepackage{enumitem} % directly use itemize, easily specify indent and everything
\setlist[itemize]{leftmargin=*,label=$\bullet$}%leftmargin=*,itemsep=0pt} %topsep=5pt
\setlist[enumerate]{label={\arabic*)}}
\usepackage{hyperref}
\usepackage{wrapfig}
\usepackage{textpos}
\usepackage{bibentry} % for publication list
\makeatletter\let\saved@bibitem\@bibitem\makeatother % make hyperref and bibentry compatible!!!
\nobibliography*
\usepackage{fancybox}% shadow for image
%\usepackage{empheq} % emphasize equations
\usepackage{bm}
\usepackage{arydshln} % for dashline in table or matrix
\linespread{1.3}
\usepackage{multimedia}

\usepackage{setspace} \setstretch{1.2}

\usepackage{framed}
\colorlet{shadecolor}{black!5}
% for box, page breakable, very good!!
\usepackage[framemethod=TikZ]{mdframed}%
\mdfdefinestyle{myFrame}{%
    linecolor=gray!15!white,%gray
    outerlinewidth=0.1pt,
    roundcorner=3pt,
    skipabove=15pt, % the space before the entire box
    skipbelow=15pt, % the space after the entire box. Please see the figure 2 in the manual, very clear!
    innertopmargin=10pt,%\baselineskip,
    innerbottommargin=10pt,%\baselineskip,
    %innerrightmargin=10pt,
    %innerleftmargin=10pt,
    splittopskip=\baselineskip,
    splitbottomskip=\baselineskip,
    backgroundcolor=gray!10!white,
    frametitlerule=true,
    frametitlebackgroundcolor=gray!20!white,
    frametitleaboveskip=5pt,
    frametitlebelowskip=5pt,
}
\mdfdefinestyle{myAlgo}{%
    linecolor=gray!100!white,%gray
    outerlinewidth=0.1pt,
    roundcorner=3pt,
    skipabove=15pt, % the space before the entire box
    skipbelow=15pt, % the space after the entire box. Please see the figure 2 in the manual, very clear!
    innertopmargin=10pt,%\baselineskip,
    innerbottommargin=10pt,%\baselineskip,
    %innerrightmargin=10pt,
    %innerleftmargin=10pt,
    splittopskip=\baselineskip,
    splitbottomskip=\baselineskip,
    backgroundcolor=gray!0!white,
    frametitlerule=true,
    frametitlebackgroundcolor=gray!20!white,
    frametitleaboveskip=5pt,
    frametitlebelowskip=5pt,
}


\usepackage{tikz}
\usetikzlibrary{calc} % for calculation functions in Tikz let, in commands in Tikz
\usetikzlibrary{shapes} % for block diagram
\usetikzlibrary{chains}
\usetikzlibrary{fit}
\usetikzlibrary{arrows}
\usetikzlibrary{decorations.text} % text along path

\newcommand{\blue}[1]{\textcolor{blue}{#1}}
\definecolor{myred}{RGB}{200,0,0}
\newcommand{\red}[1]{\textcolor{myred}{#1}} %magenta purple
\newcommand{\I}{\mathcal{I}}
\newcommand{\tr}{\mathrm{tr}}
\newcommand{\Null}{\mathrm{Null}}
\newcommand{\Range}{\mathrm{Range}}
\newcommand{\one}{\mathbf{1}}
\newcommand{\rank}{\mathrm{rank}}
\newcommand{\myspan}{\mathrm{span}}
\newcommand{\mydiag}{\mathrm{diag}}
\newcommand{\D}{\mathrm{d}}
\renewcommand{\d}{\mathrm{d}}
\newcommand{\blkdiag}{\mathrm{blkdiag}}
\newcommand{\sgn}{\mathrm{sgn}}
\newcommand{\T}{\mathrm{T}}
\newcommand{\myqed}{\hfill$\blacksquare$}
\newcommand{\ep}{\varepsilon}
\newcommand{\sig}{\mathrm{sig}_a}
%\newcommand{\sigep_}[1]{\sig(\ep_{#1})}
\newcommand{\R}{\mathbb{R}}
\newcommand{\A}{\mathcal{A}}
\newcommand{\G}{\mathcal{G}}
\newcommand{\E}{\mathbb{E}}
\newcommand{\X}{\mathcal{X}}
\newcommand{\V}{\mathcal{V}}
\newcommand{\N}{\mathcal{N}}
\newcommand{\M}{\mathcal{M}}
\renewcommand{\H}{\mathcal{H}}
\renewcommand{\L}{\mathcal{B}}
\renewcommand{\S}{\mathcal{S}}
\newcommand{\xe}{x_{\text{e}}}
%\newcommand{\Null}[1]{\mathrm{Null}\left(#1\right)}
\newcommand{\sk}[1]{\left[#1\right]_\times} % skew symmetric operator
\newcommand{\dia}[1]{\mathrm{diag}\left(#1\right)} % block diagnal matrix
%\renewcommand{\span}[1]{\mathrm{span}\left\{#1\right\}} % ERROR when redefine \span
\newcommand{\Var}{\mathrm{Var}}
\newcommand{\var}{\mathrm{var}}


\graphicspath{{figures/}}

% for tikz, theorem, lemma ... environments have already been declared. You don't need to declare, or you need to use other names than theorem or lemma, such as my_theorem.
%\newtheorem{my_theorem}{Theorem}
%\newtheorem{my_lemma}{Lemma}
\newtheorem{assumption}{Assumption} % necessary for beamer
%\newtheorem{my_remark}{Remark}
\newtheorem{proposition}{Proposition} % necessary for beamer
%\newtheorem{my_corollary}{Corollary}
%\newtheorem{my_example}{Example}
%\newtheorem{my_definition}{Definition}
%\newtheorem{my_problem}{Problem}

%##################################################
\newcommand{\pagetitle}[1]{\textbf{\textcolor{BlueViolet}{$\circ$ #1}}} %!!!
\newcommand{\pagehighlight}[1]{\textbf{\textcolor{Brown}{#1}}} %!!!
\newcommand{\mypause}{\pause} % this is useful for slide show. if you don't want pause any more, just set it as blank
%\newcommand{\mybullet}{\textcolor{BlueViolet}{$\blacksquare$} }%{$\rhd$ }
%\newcommand{\myhighsign}{$\star$ }% the sign to highlight a sentence

%##################################################
% To highlight equation. Example: \begin{align*} \boxed{xxx} \end{align*}
% does not support multiline equations
% put color to \boxed math command
\newcommand*{\boxcolor}{gray}
\makeatletter
\renewcommand{\boxed}[1]{\textcolor{\boxcolor}{%
%\tikz[baseline={([yshift=-1ex]current bounding box.center)}] \node [rectangle, minimum width=1ex,rounded corners,draw] {\normalcolor\m@th$\displaystyle#1$};}}
\tikz[baseline={([yshift=-1ex]current bounding box.center)}] \node [rectangle, minimum width=2ex,rounded corners,draw] {\normalcolor\m@th$\displaystyle#1$};}}
\makeatother

%##################################################
% set my own theme
\def\structureHeight{9mm}
\usetheme[height=\structureHeight]{Rochester}
\usecolortheme[RGB={0,0,128}]{structure}
\setbeamertemplate{items}[circle]%rectangle, triangle,circle
\setbeamertemplate{blocks}[rounded][shadow=true]
\setbeamertemplate{navigation symbols}{}
%\addtobeamertemplate{frametitle}{} % specify the logo
%{
%    \begin{textblock*}{100mm}(.87\textwidth,-\structureHeight)
%        \includegraphics[height=6.6mm,width=3cm,keepaspectratio]{../common_figures_private/westlake_logo.png} % add logo
%    \end{textblock*}
%}
\addtobeamertemplate{frametitle}{\vskip4pt}{} % specify
%\setbeamerfont{frametitle}{size=\large}
\definecolor{mylightgray}{RGB}{240 240 240}
\definecolor{mykhaki}{RGB}{240 230 140}% khaki color
\definecolor{mylightYellow}{RGB}{255,255,224} % light yellow
%\setbeamercolor{beamercolor1}{bg=mylightgray, fg=black}
%\setbeamercolor{beamercolor2}{bg=mylightYellow,fg=black}%{bg=yellow!90!white, fg=black}
% background and foreground color
\setbeamercolor{background canvas}{bg=black!0!white} % background color of every slide! My previous value was black!10!white for my lecture videos!
\setbeamercolor{normal text}{bg=black!10!white} % background color for e.g. theorem environment. When canvas is 10, here it can be 20; the bg for normal text changes the color of hidden text when you use overlay

\setbeamercolor{block title}{bg=mykhaki,fg=black}
\defbeamertemplate{footline}{zsy_frameNumber}
{%
  \hspace{5pt}  \emph{Shiyu Zhao}
  \hspace*{\fill}%
  \usebeamercolor[fg]{page number in head/foot}%
  \insertframenumber\,/\,\inserttotalframenumber \vspace{0pt} \hspace{5pt}
  \vskip5pt
}
\setbeamertemplate{footline}[zsy_frameNumber]
%##################################################
\setbeamercovered{transparent=0} % a good value for transparent text is 20
% when using the overlay commands like \onslide or \uncover, the text will NOT be invisible, instead it will be like transparent
%\pause will also have the transparent effect: command \pause is easy to use: to make it invisible, change the value to zero.



\usepackage{multimedia}
\linespread{1.3}
\newcommand{\hl}{$\triangleright$ }
%\newcommand{\blue}[1]{\textcolor{blue}{#1}}
%\newcommand{\red}[1]{\textcolor{red}{#1}}

\begin{document}

%%%%%%%%%%%%%%%%%%%%%%%%%%%%%%%%%%%%%%%%%%%%%%%%%%%%%%%%%%%%%%%%%%%%%%%%%%%%%%%%%
% define the author, date etc. information
%\subtitle{Mathematical and Biological Foundation for Reinforcement Learning}
\title{Lecture 2: State Value and Bellman Equation}

\author{Shiyu Zhao
\newline
\newline {\small Department of Artificial Intelligence}
\newline {\small Westlake University}
}
%\logo{\includegraphics[width=1cm,height=1cm,keepaspectratio]{NUSLogo.png}~}
%\date{\today}
\date{}
\subject{}


%%%%%%%%%%%%%%%%%%%%%%%%%%%%%%%%%%%%%%%%%%%%%%%%%%%%%%%%%%%%%%%%%%%%%%%%%%%%%%%%%

{
\setbeamertemplate{footline}{} % remove the frame number of the title page
\begin{frame}
    %\frametitle{Lecture: Networked Dynamic Systems}
    \addtocounter{framenumber}{-1} % discounter the title page, otherwise the frame number starts from 2 instead of 1
    \titlepage % this only gives the author etc. information
\end{frame}
}


%------------------------------------------
\begin{frame}
\frametitle{Outline}
\begin{figure}[h]
  \centering
\includegraphics[width=0.8\linewidth]{Figure_chapterRelationship.pdf}
\end{figure}
\end{frame}
%------------------------------------------

\begin{frame}
\frametitle{Outline}
\tableofcontents
\end{frame}
\AtBeginSection[]% put it to the start of each section
{
  \begin{frame}
    \frametitle{Outline}
    \tableofcontents[currentsection]
  \end{frame}
}
%%--------------------------------------
%\AtBeginSection[]% put it to the start of each section
%{
%  \begin{frame}
%    \frametitle{Outline}
%    \tableofcontents[currentsection]
%  \end{frame}
%}
%\section{Revisit the last lecture}
%%--------------------------------------
%\begin{frame}
%\frametitle{Revisit}
%\begin{definition}[Optimal policy]
%A policy $\pi^*$ is optimal if $v_{\pi^*}(s)\ge v_{\pi}(s)$ for all $s$ and for any other policy $\pi$.
%\end{definition}
%\pause
%\hl Why we care about optimal policies? RL's aim is to find optimal policies!!
%
%\pause
%\hl Questions we asked:
%\begin{itemize}
%\item Does the optimal policy exist?
%\item Is the optimal policy unique?
%\item How to obtain the optimal policy?
%\item Is the optimal policy stochastic or deterministic?
%\end{itemize}
%\hl With the Bellman optimality equation, we are able to answer these questions.
%
%\end{frame}
%%--------------------------------------
%\begin{frame}
%\frametitle{Revisit}
%
%The Bellman optimality equation is
%\begin{align*}
%v=\max_{\pi}(r_\pi+\gamma  P_\pi v)
%\end{align*}
%
%\pause
%Based on the \textcolor[rgb]{0.00,0.07,1.00}{contraction mapping theorem}, we have the following results.
%\begin{itemize}
%\item \emph{Existence and Uniqueness}: exist a unique solution $v^*$
%\begin{align*}
%v^*=\max_{\pi}(r_\pi+\gamma  P_\pi v^*)
%\end{align*}
%\item \emph{Optimal policy}: The policy
%\begin{align*}
%\pi^*=\arg\max_{\pi}(r_\pi+\gamma  P_\pi v^*)
%\end{align*}
%leads to
%\begin{align*}
%v^*=r_{\pi^*}+\gamma  P_{\pi^*} v^*
%\end{align*}
%Hence the state value of $\pi^*$ is $v_{\pi^*}=v^*$.
%\item \emph{Optimality}: $\pi^*$ is optimal because $v_{\pi^*}\ge v_\pi$ for any $\pi$.
%\end{itemize}
%\end{frame}
%--------------------------------------
\AtBeginSection[]% put it to the start of each section
{
  \begin{frame}
    \frametitle{Outline}
    \tableofcontents[currentsection]
  \end{frame}
}
\section{Value iteration algorithm}
\begin{frame}
\frametitle{Value iteration algorithm}

\hl How to solve the Bellman optimality equation?
$$v=f(v)=\max_\pi (r_\pi+\gamma P_{\pi}v)$$

\pause
\hl The \textbf{contraction mapping theorem} suggests an iterative algorithm:
$$v_{k+1}=f(v_k)=\max_\pi(r_\pi+\gamma  P_\pi v_k),\quad k=0,1,2,3\dots$$
where $v_0$ can be arbitrary.
This algorithm can eventually find the optimal state value and an optimal policy.

\vspace{10pt}
\pause
\hl This algorithm is called the \blue{value iteration} algorithm!

\hl We next study the \textbf{implementation details} of this algorithm.
\end{frame}
%--------------------------------------
\begin{frame}
\frametitle{Value iteration algorithm}

The algorithm (matrix-vector form)
$$v_{k+1}= f(v_k)=\max_\pi(r_\pi+\gamma  P_\pi v_k),\quad k=0,1,2,3\dots$$
can be decomposed to two steps.
\pause
\begin{itemize}
\item \blue{Step 1: policy update.} This step is to solve $$\pi_{k+1}=\arg\max_\pi (r_\pi +\gamma  P_\pi v_k)$$
where $v_k$ is given.
\pause
\item \blue{Step 2: value update.} This step is to compute the following equation:
$$v_{k+1}=r_{\pi_{k+1}} +\gamma  P_{\pi_{k+1}} v_k$$
\end{itemize}
\pause
\textbf{Question:} is $v_k$ a state value?

\textbf{Answer:} No, because $v_k$ may not satisfy any Bellman equation.
\end{frame}
%--------------------------------------
\begin{frame}
\frametitle{Value iteration algorithm}

\hl We next study the elementwise form in order to implement the algorithm.

\begin{itemize}
\item Matrix-vector form is useful for \textbf{theoretical analysis}.

\item Elementwise form is useful for \textbf{implementation}.
\end{itemize}
\end{frame}
%--------------------------------------
\begin{frame}
\frametitle{Value iteration algorithm - Elementwise form}

\hl \blue{Step 1: Policy update}

The elementwise form of
$$\pi_{k+1}=\arg\max_\pi (r_\pi +\gamma  P_\pi v_k)$$
is
{\small
\begin{align*}
\pi_{k+1}\blue{(s)}=\arg\max_\pi \sum_{a}{\pi(a|s)}\underbrace{\left(\sum_{r}p(r|s,a)r + \gamma \sum_{s'}p(s'|s,a)\red{v_k(s')}\right)}_{\red{q_k(s,a)}},\quad s\in\mathcal{S}
\end{align*}
}

\pause
The optimal policy solving the above optimization problem is
\begin{align*}%\label{eq_optimalPolicyDeterministic}
\blue{\pi_{k+1}(a|s)=
\left\{
  \begin{array}{ll}
   1 & a=a^*_k(s) \\
   0 & a\ne a^*_k(s) \\
  \end{array}
\right.}
\end{align*}
where $\blue{a^*_k(s)=\arg\max_a q_k(a,s)}$.
$\pi_{k+1}$ is called a \textbf{greedy policy}, since it simply selects the greatest q-value.

\end{frame}
%--------------------------------------
\begin{frame}
\frametitle{Value iteration algorithm - Elementwise form}

\hl \blue{Step 2: Value update}

The elementwise form of
$$v_{k+1}=r_{\pi_{k+1}} +\gamma  P_{\pi_{k+1}} v_k$$
is
\begin{align*}
v_{k+1}(s)=\sum_{a}\red{\pi_{k+1}(a|s)}\underbrace{\left(\sum_{r}p(r|s,a)r + \gamma \sum_{s'}p(s'|s,a)\red{v_k(s')}\right)}_{\red{q_k(s,a)}},\quad s\in\mathcal{S}
\end{align*}

\pause
Since $\pi_{k+1}$ is greedy, the above equation is simply
\begin{align*}
\blue{v_{k+1}(s)=\max_a q_k(a,s)}
\end{align*}

\end{frame}
%--------------------------------------
\begin{frame}
\frametitle{Value iteration algorithm - Pseudocode}
%\vspace{-10pt}
\hl Procedure summary:
\begin{align*}
{\small\blue{
v_k(s)\rightarrow q_k(s,a) \rightarrow \text{greedy policy } \pi_{k+1}(a|s) \rightarrow \text{new value } v_{k+1}=\max_a q_k(s,a)}}
\end{align*}

\pause

\begin{figure}[t]
{\small
\begin{mdframed}[style=myAlgo,nobreak=true,frametitle={Pseudocode: Value iteration algorithm}]
{\fontfamily{cmss}\selectfont
\textbf{Initialization:} The probability models $p(r|s,a)$ and $p(s'|s,a)$ for all $(s,a)$ are known. Initial guess $v_0$.

\textbf{Aim:} Search for the optimal state value and an optimal policy by solving the Bellman optimality equation.

\vspace{5pt}
While $v_k$ has not converged in the sense that $\|v_k-v_{k-1}\|$ is greater than a predefined small threshold, for the $k$th iteration, do

\qquad For every state $s\in\S$, do

\qquad\qquad For every action $a\in\A(s)$, do

\qquad\qquad\qquad q-value: $q_k(s,a)=\sum_{r}p(r|s,a)r + \gamma \sum_{s'}p(s'|s,a)v_k(s')$

\qquad\qquad Maximum action value: $a^*_k(s)=\arg\max_a q_k(a,s)$

\qquad\qquad \emph{Policy update:} $\pi_{k+1}(a|s)=1$ if $a=a^*_k$, and $\pi_{k+1}(a|s)=0$ otherwise

\qquad\qquad \emph{Value update:} $v_{k+1}(s)=\max_a q_k(a,s)$
}%\fontfamily
\end{mdframed}}
\end{figure}

\end{frame}

%
%%--------------------------------------
%\begin{frame}
%\frametitle{Example}
%
%\begin{figure}[h]
%  \centering
%  \includegraphics[width=0.4\linewidth]{fig_1D3Grid}
%\end{figure}
%
% Example: Manually solve the BOE.
%
%\begin{itemize}
%\item Why manually? Can understand better.
%\item Why so simple example? Can be calculated manually.
%\end{itemize}
%
%Actions: $a_\ell, a_0, a_r$ represent go left, stay unchanged, and go right.
%
%Reward: entering the target area: +1; try to go out of boundary -1.
%
%\end{frame}
%
%%--------------------------------------
%\begin{frame}
%\frametitle{Example}
%\begin{figure}[h]
%  \centering
%  \includegraphics[width=0.4\linewidth]{fig_1D3Grid}
%\end{figure}
%
%\pause
% The values of $q(s,a)$
%\begin{table}[h]
%\centering
%%\resizebox{0.8\columnwidth}{!}{
%\begin{tabular}{|c|c|c|c|}%{|c|p{2cm}|p{2cm}|p{2cm}|}
%  % after \\: \hline or \cline{col1-col2} \cline{col3-col4} ...
%  \hline
%  q-value table & \blue{$a_\ell$} & \blue{$a_0$} & \blue{$a_r$} \\
%  \hline
%  \blue{$s_1$} & $-1+\gamma v(s_1)$ & $0+\gamma v(s_1)$ & $1+\gamma v(s_2)$\\
%  \hline
%  \blue{$s_2$} & $0+\gamma v(s_1)$ & $1+\gamma v(s_2)$ & $0+\gamma v(s_3)$\\
%  \hline
%  \blue{$s_3$} & $1+\gamma v(s_2)$ & $0+\gamma v(s_3)$ & $-1+\gamma v(s_3)$\\
%  \hline
%\end{tabular}
%%}
%%\caption{The value of $q(a,s)$}
%\end{table}
%Consider $\gamma=0.9$
%
%%\pause
%% Question: is this action value? Very similar but not, because $v$ is not state value. We call it q-value.
%
%\end{frame}
%%--------------------------------------
%\begin{frame}
%\frametitle{Example}
% Our objective is to find $v^*(s_i)$ and $\pi^*$
%
%\pause
% $k=0$:
%
%v-value: select $v_0(s_1)=v_0(s_2)=v_0(s_3)=0$
%
%q-value (using the previous table):
%\begin{table}[h]
%\centering
%%\resizebox{0.8\columnwidth}{!}{
%\begin{tabular}{|c|c|c|c|}%{|c|p{2cm}|p{2cm}|p{2cm}|}
%  % after \\: \hline or \cline{col1-col2} \cline{col3-col4} ...
%  \hline
%   & \blue{$a_\ell$} & \blue{$a_0$} & \blue{$a_r$} \\
%  \hline
%  \blue{$s_1$} & $-1$ & $0$ & $1$\\
%  \hline
%  \blue{$s_2$} & $0$ & $1$ & $0$\\
%  \hline
%  \blue{$s_3$} & $1$ & $0$ & $-1$\\
%  \hline
%\end{tabular}
%%}
%%\caption{The value of $q(a,s)$ at $k=0$}
%\end{table}
%\pause
%Greedy policy (select the greatest q-value)
%\begin{align*}
%\pi(a_r|s_1)=1,\quad \pi(a_0|s_2)=1,\quad \pi(a_\ell|s_3)=1
%\end{align*}
%v-value: $v_{1}(s)=\max_a q_0(s,a)$
%$$v_1(s_1)=v_1(s_2)=v_1(s_3)=1$$
%
%This this policy good? Yes!
%\end{frame}
%%--------------------------------------
%\begin{frame}
%\frametitle{Example}
%
%\begin{itemize}
%
%\item $k=1$:
%
%Excise: With $v_1(s)$ calculated in the last step, calculate by yourself.
%\pause
%q-value:
%\begin{table}[h!]
%\centering
%%\resizebox{0.5\columnwidth}{!}{
%\begin{tabular}{|c|c|c|c|}%{|c|p{2cm}|p{2cm}|p{2cm}|}
%  % after \\: \hline or \cline{col1-col2} \cline{col3-col4} ...
%  \hline
%   & \blue{$a_\ell$} & \blue{$a_0$} & \blue{$a_r$} \\
%  \hline
%  \blue{$s_1$} & $-0.1$ & $0.9$ & $1.9$\\
%  \hline
%  \blue{$s_2$} & $0.9$ & $1.9$ & $0.9$\\
%  \hline
%  \blue{$s_3$} & $1.9$ & $0.9$ & $-0.1$\\
%  \hline
%\end{tabular}
%%}
%%\caption{The value of $q(a,s)$ at $k=1$}
%\end{table}
%\pause
%Greedy policy (select the greatest q-value):
%\begin{align*}
%\pi(a_r|s_1)=1,\quad \pi(a_0|s_2)=1,\quad \pi(a_\ell|s_3)=1
%\end{align*}
%The policy is the same as the previous one, which is already optimal.
%
%v-value: $v_2(s)=...$
%\item $k=2,3,\dots$
%\end{itemize}
%\end{frame}

%--------------------------------------
\begin{frame}
\frametitle{Value iteration algorithm - Example}

\hl The reward setting is $r_{\rm boundary}=r_{\rm forbidden}=-1$, $r_{\rm target}=1$. The discount rate is $\gamma=0.9$.

\begin{figure}[h]
  \centering
  \includegraphics[width=0.7\linewidth]{fig_ValueIterationDemoGridExample}
  %\caption{An example to demonstrate the implementation of the value iteration algorithm.}
  %\label{figChapterVIPI_demoVI}
\end{figure}

\pause
q-table: The expression of $q(s,a)=\sum_{r}p(r|s,a)r + \gamma \sum_{s'}p(s'|s,a)\red{v_k(s')}$:
\begin{table}[t]
\centering
\resizebox{\columnwidth}{!}{
\begin{tabular}{|c|c|c|c|c|c|}
  % after \\: \hline or \cline{col1-col2} \cline{col3-col4} ...
  \hline
  q-value & \blue{$a_1$} & \blue{$a_2$} & \blue{$a_3$} & \blue{$a_4$} & \blue{$a_5$} \\
  \hline
  \blue{$s_1$} & $-1+\gamma v(s_1)$ & $-1+\gamma v(s_2)$ & $0+\gamma v(s_3)$ & $-1+\gamma v(s_1)$ & $0+\gamma v(s_1)$\\
  \hline
  \blue{$s_2$} & $-1+\gamma v(s_2)$ & $-1+\gamma v(s_2)$ & $1+\gamma v(s_4)$ & $0+\gamma v(s_1)$ & $-1+\gamma v(s_2)$\\
  \hline
  \blue{$s_3$} & $0+\gamma v(s_1)$ & $1+\gamma v(s_4)$ & $-1+\gamma v(s_3)$ & $-1+\gamma v(s_3)$ & $0+\gamma v(s_3)$\\
  \hline
  \blue{$s_4$} & $-1+\gamma v(s_2)$ & $-1+\gamma v(s_4)$ & $-1+\gamma v(s_4)$ & $0+\gamma v(s_3)$ & $1+\gamma v(s_4)$\\
  \hline
\end{tabular}
}
%\caption{The expression of $q(s,a)$.}
%\label{table_qvalueExpressionForDemoVI}
\end{table}

\end{frame}
%--------------------------------------
\begin{frame}
\frametitle{Value iteration algorithm - Example}

\begin{itemize}
\item $k=0$: let $v_0(s_1)=v_0(s_2)=v_0(s_3)=v_0(s_4)=0$
{\small
\begin{table}[t!]
\centering
%\resizebox{0.9\columnwidth}{!}{
\begin{tabular}{|c|c|c|c|c|c|}
  % after \\: \hline or \cline{col1-col2} \cline{col3-col4} ...
  \hline
  q-value & \blue{$a_1$} & \blue{$a_2$} & \blue{$a_3$} & \blue{$a_4$} & \blue{$a_5$} \\
  \hline
  \blue{$s_1$} & $-1$ & $-1$ & $\red{0}$ & $-1$ & $\red{0}$\\
  \hline
  \blue{$s_2$} & $-1$ & $-1$ & $\red{1}$ & $0$ & $-1$\\
  \hline
  \blue{$s_3$} & $0$ & $\red{1}$ & $-1$ & $-1$ & $0$\\
  \hline
  \blue{$s_4$} & $-1$ & $-1$ & $-1$ & $0$ & $\red{1}$\\
  \hline
\end{tabular}
%}
\end{table}
}

\pause
\blue{Step 1: Policy update:}
\begin{align*}
\pi_1(a_5|s_1)=1, \quad  \pi_1(a_3|s_2)=1, \quad \pi_1(a_2|s_3)=1, \quad \pi_1(a_5|s_4)=1
\end{align*}

\pause
\blue{Step 2: Value update:}
$$v_1(s_1)=0, \quad v_1(s_2)=1, \quad v_1(s_3)=1, \quad v_1(s_4)=1$$

This policy is visualized in Figure (b).
\end{itemize}

\end{frame}
%--------------------------------------
\begin{frame}
\frametitle{Value iteration algorithm - Example}
\begin{itemize}
\item $k=1$: since $v_1(s_1)=0, v_1(s_2)=1, v_1(s_3)=1, v_1(s_4)=1$, we have

\pause

\begin{table}[h!]
\centering
\resizebox{0.8\columnwidth}{!}{
\begin{tabular}{|c|c|c|c|c|c|}
  % after \\: \hline or \cline{col1-col2} \cline{col3-col4} ...
  \hline
  q-table & \blue{$a_1$} & \blue{$a_2$} & \blue{$a_3$} & \blue{$a_4$} & \blue{$a_5$} \\
  \hline
  \blue{$s_1$} & $-1+\gamma0$ & $-1+\gamma 1$ & $\red{0+\gamma 1}$ & $-1+\gamma 0$ & $0+\gamma 0$\\
  \hline
  \blue{$s_2$} & $-1+\gamma 1$ & $-1+\gamma 1$ & $\red{1+\gamma 1}$ & $0+\gamma 0$ & $-1+\gamma 1$\\
  \hline
  \blue{$s_3$} & $0+\gamma 0$ & $\red{1+\gamma 1}$ & $-1+\gamma 1$ & $-1+\gamma 1$ & $0+\gamma 1$\\
  \hline
  \blue{$s_4$} & $-1+\gamma 1$ & $-1+\gamma 1$ & $-1+\gamma 1$ & $0+\gamma 1$ & $\red{1+\gamma 1}$\\
  \hline
\end{tabular}
}
%\caption{The value of $q(a,s)$ at $k=1$}
\end{table}

\pause
\blue{Step 1: Policy update:}
\vspace{-5pt}
$$\pi_2(a_3|s_1)=1, \ \pi_2(a_3|s_2)=1, \ \pi_2(a_2|s_3)=1, \ \pi_2(a_5|s_4)=1.$$

\pause
\blue{Step 2: Value update:}
\vspace{-5pt}
$$v_2(s_1)=\gamma1, \ v_2(s_2)=1+\gamma1, \ v_2(s_3)=1+\gamma1, \ v_2(s_4)=1+\gamma1.$$
This policy is visualized in Figure (c).
The policy is already optimal!!
\pause
\item $k=2,3,\dots$. Stop when $\|v_k-v_{k+1}\|$ is smaller than a predefined threshold.
\end{itemize}
\end{frame}
%--------------------------------------
\AtBeginSection[]% put it to the start of each section
{
  \begin{frame}
    \frametitle{Outline}
    \tableofcontents[currentsection]
  \end{frame}
}
\section{Policy iteration algorithm}
\begin{frame}
\frametitle{Policy iteration algorithm}

\hl Algorithm description:

Given a random \blue{initial policy} $\pi_0$,

\pause
\begin{itemize}
\item \blue{Step 1: policy evaluation (PE)}

This step is to calculate the state value of $\pi_k$:
$$v_{\pi_k}=r_{\pi_k}+\gamma  P_{\pi_k} v_{\pi_k}$$
Note that $v_{\pi_k}$ is a state value function.
\pause
\item \blue{Step 2: policy improvement (PI)}
\begin{align*}
\pi_{k+1}=\arg\max_{\pi} (r_\pi +\gamma  P_\pi v_{\pi_k})
\end{align*}
The maximization is componentwise!
\end{itemize}

\pause
Similar to the value iteration algorithm? Be patient. We will compare them later.
\end{frame}
%--------------------------------------
\begin{frame}
\frametitle{Policy iteration algorithm}
\hl The algorithm leads to a sequence:
$$\pi_0\xrightarrow{PE} v_{\pi_0}\xrightarrow{PI} \pi_1 \xrightarrow{PE} v_{\pi_1}\xrightarrow{PI} \pi_2 \xrightarrow{PE} v_{\pi_2}\xrightarrow{PI} \dots$$
PE=policy evaluation, PI=policy improvement

\pause
\vspace{10pt}
\hl Questions:
\begin{itemize}
\item Q1: In the policy evaluation step, how to get the state value $v_{\pi_k}$ by solving the Bellman equation?
\pause
\item Q2: In the policy improvement step, why is the new policy $\pi_{k+1}$ better than $\pi_k$?
\pause
\item Q3: Why can such an iterative algorithm finally reach an optimal policy?
\pause
\item Q4: What is the relationship between this policy iteration algorithm and the previous value iteration algorithm?
\end{itemize}


\end{frame}
%--------------------------------------
\begin{frame}
\frametitle{Policy iteration algorithm}

\hl \red{Q1: In the policy evaluation step, how to get the state value $v_{\pi_k}$ by solving the Bellman equation?}
$$v_{\pi_k}=r_{\pi_k}+\gamma  P_{\pi_k} v_{\pi_k}$$
\vspace{-20pt}
\pause
\begin{itemize}
\item Closed-form solution:
$$v_{\pi_k}=(I-\gamma P_{\pi_k})^{-1}r_{\pi_k}$$

\item Iterative solution:
\begin{align*}%\label{eq_iterativeAlgoPEStepofPI}
v_{\pi_k}^{(j+1)}=r_{\pi_k}+\gamma  P_{\pi_k} v_{\pi_k}^{(j)},\quad j=0,1,2,...
\end{align*}
\end{itemize}

Already studied in the lecture about Bellman equation.

\pause
\vspace{10pt}
\blue{\hl Policy iteration is an iterative algorithm with another iterative algorithm embedded in the policy evaluation step!}

\end{frame}
%--------------------------------------
\begin{frame}
\frametitle{Policy iteration algorithm}

\hl \red{Q2: In the policy improvement step, why is the new policy $\pi_{k+1}$ better than $\pi_k$?}

\pause
\begin{lemma}[Policy Improvement]\label{lemma_PIAlgoPolicyImprove}
If $\pi_{k+1}=\arg\max_{\pi} (r_\pi +\gamma  P_\pi v_{\pi_k})$, then $v_{\pi_{k+1}}\ge v_{\pi_{k}}$ for any $k$.
\end{lemma}

See the proof in the book.

\end{frame}
%--------------------------------------
\begin{frame}
\frametitle{Policy iteration algorithm}

\hl \red{Q3: Why can such an iterative algorithm finally reach an optimal policy?}

\pause
Since every iteration would improve the policy, we know
$$v_{\pi_0}\le v_{\pi_1}\le v_{\pi_2}\le\dots\le v_{\pi_{k}} \le \dots\le v^*.$$
As a result, $v_{\pi_k}$ keeps \textbf{increasing} and will converge.

\pause
\vspace{10pt}
Still need to prove that it converges to the optimal value $v^*$.

\begin{theorem}[Convergence of Policy Iteration]
The state value sequence $\{v_{\pi_k}\}_{k=0}^\infty$ generated by the policy iteration algorithm converges to the optimal state value $v^*$. As a result, the policy sequence $\{\pi_k\}_{k=0}^\infty$ converges to an optimal policy.
\end{theorem}

The proof is given in my book.

\end{frame}
%--------------------------------------
\begin{frame}
\frametitle{Policy iteration algorithm}


\hl \red{Q4: What is the relationship between policy iteration and value iteration algorithms?}

Will be explained in detail later.

\end{frame}
%--------------------------------------
\begin{frame}
\frametitle{Policy iteration algorithm - Elementwise form}

\blue{Step 1: Policy evaluation}

\pause

\hl Matrix-vector form: $v_{\pi_k}^{(j+1)}=r_{\pi_k}+\gamma P_{\pi_k}v_{\pi_k}^{(j)},\quad j=0,1,2,\dots$

\pause
\hl Elementwise form:
\begin{align*}
v_{\pi_k}^{(j+1)}(s)=\sum_{a}{\pi_k(a|s)}\left(\sum_{r}p(r|s,a)r + \gamma \sum_{s'}p(s'|s,a)\red{v_{\pi_k}^{(j)}(s')}\right),\quad s\in\mathcal{S}
\end{align*}
Stop when $j$ is sufficiently large or $\|v_{\pi_k}^{(j+1)}-v_{\pi_k}^{(j)}\|$ is sufficiently small.
\end{frame}
%--------------------------------------
\begin{frame}
\frametitle{Policy iteration algorithm - Elementwise form}
\blue{Step 2: Policy improvement}

\pause
\hl Matrix-vector form: $\pi_{k+1}=\arg\max_{\pi} (r_\pi +\gamma  P_\pi \red{v_{\pi_k}})$

\pause
\hl Elementwise form
{\footnotesize
\begin{align*}
\pi_{k+1}\blue{(s)}=\arg\max_\pi \sum_{a}{\pi(a|s)}\underbrace{\left(\sum_{r}p(r|s,a)r + \gamma \sum_{s'}p(s'|s,a)\red{v_{\pi_k}(s')}\right)}_{q_{\pi_k}(s,a)},\quad s\in\mathcal{S}.
\end{align*}
}

Here, $q_{\pi_k}(s,a)$ is the action value under policy $\pi_k$.
Let
$$a^*_k(s)=\arg\max_a q_{\pi_k}(a,s)$$
Then, the greedy policy is
\begin{align*}%\label{eq_optimalPolicyDeterministic}
{\pi_{k+1}(a|s)=
\left\{
  \begin{array}{ll}
   1 & a=a^*_k(s), \\
   0 & a\ne a^*_k(s). \\
  \end{array}
\right.}
\end{align*}

\end{frame}
%--------------------------------------
\begin{frame}
\frametitle{Policy iteration algorithm - Implementation}
\footnotesize
\begin{figure}[t]
\begin{mdframed}[style=myAlgo,nobreak=true,frametitle={Pseudocode: Policy iteration algorithm}]
{\fontfamily{cmss}\selectfont
\textbf{Initialization:} The probability models $p(r|s,a)$ and $p(s'|s,a)$ for all $(s,a)$ are known. Initial guess $\pi_0$.

\textbf{Aim:} Search for the optimal state value and an optimal policy.

While $v_{\pi_k}$ has not converged, for the $k$th iteration, do

\qquad \blue{\emph{Policy evaluation:}}

\qquad Initialization: an arbitrary initial guess $v_{\pi_k}^{(0)}$

\qquad While $v_{\pi_k}^{(j)}$ has not converged, for the $j$th iteration, do

\qquad\qquad For every state $s\in\S$, do

\qquad\qquad\qquad $v_{\pi_k}^{(j+1)}(s)=\sum_{a}{\pi_k(a|s)}\left[\sum_{r}p(r|s,a)r + \gamma \sum_{s'}p(s'|s,a) {v_{\pi_k}^{(j)}(s')}\right]$

\qquad \blue{\emph{Policy improvement:}}

\qquad For every state $s\in\S$, do

\qquad\qquad For every action $a\in\A$, do

\qquad\qquad\qquad $q_{\pi_k}(s,a)=\sum_{r}p(r|s,a)r + \gamma \sum_{s'}p(s'|s,a)v_{\pi_k}(s')$

\qquad\qquad $a^*_k(s)=\arg\max_a q_{\pi_k}(s,a)$

\qquad\qquad $\pi_{k+1}(a|s)=1$ if $a=a^*_k$, and $\pi_{k+1}(a|s)=0$ otherwise
}
\end{mdframed}
\end{figure}
\end{frame}
%--------------------------------------
\begin{frame}
\frametitle{Policy iteration algorithm - Simple example}

\begin{figure}[h]
  \centering
  \includegraphics[width=0.55\linewidth]{fig_demoPIAlgo1D2Grid}
  %\caption{An example to illustrate the implementation of the policy iteration algorithm.}
  %\label{figChapterVIPI_demoPI}
\end{figure}

\begin{itemize}
\item The reward setting is $r_{\rm boundary}=-1$ and $r_{\rm target}=1$. The discount rate is $\gamma=0.9$.
\item Actions: $a_\ell, a_0, a_r$ represent going left, staying still, and going right, respectively.
\item Goal: use the policy iteration algorithm to find out an optimal policy.
\end{itemize}
\end{frame}
%--------------------------------------
\begin{frame}
\frametitle{Policy iteration algorithm - Simple example}
\hl Iteration $k=0$: \blue{Step 1: policy evaluation}

$\pi_0$ is selected as the policy in Figure (a). The Bellman equation is
\vspace{-5pt}
\begin{align*}
v_{\pi_0}(s_1)&=-1+\gamma v_{\pi_0}(s_1),\\
v_{\pi_0}(s_2)&=0+\gamma v_{\pi_0}(s_1).
\end{align*}
\vspace{-20pt}
\pause
\begin{itemize}
\item Solve the equations directly:
$$v_{\pi_0}(s_1)=-10,\quad v_{\pi_0}(s_2)=-9.$$
\pause
\vspace{-15pt}
\item Solve the equations iteratively. Select the initial guess as $v_{\pi_0}^{(0)}(s_1)=v_{\pi_0}^{(0)}(s_2)=0$:
{\footnotesize
\begin{align*}
&\left\{
  \begin{array}{l}
    v_{\pi_0}^{(1)}(s_1)=-1+\gamma v_{\pi_0}^{(0)}(s_1)=-1, \\
    v_{\pi_0}^{(1)}(s_2)=0+\gamma v_{\pi_0}^{(0)}(s_1)=0, \\
  \end{array}
\right.\\
&\left\{
  \begin{array}{l}
    v_{\pi_0}^{(2)}(s_1)=-1+\gamma v_{\pi_0}^{(1)}(s_1)=-1.9, \\
    v_{\pi_0}^{(2)}(s_2)=0+\gamma v_{\pi_0}^{(1)}(s_1)=-0.9, \\
  \end{array}
\right.\\
&\left\{
  \begin{array}{l}
    v_{\pi_0}^{(3)}(s_1)=-1+\gamma v_{\pi_0}^{(2)}(s_1)=-2.71, \\
    v_{\pi_0}^{(3)}(s_2)=0+\gamma v_{\pi_0}^{(2)}(s_1)=-1.71, \\
  \end{array}
\right.\\
&\qquad\dots
\end{align*}
}
\end{itemize}

\end{frame}
%--------------------------------------
\begin{frame}
\frametitle{Policy iteration algorithm - Simple example}
\hl Iteration $k=0$: \blue{Step 2: policy improvement}

The expression of $q_{\pi_k}(s,a)$:
{\footnotesize
\begin{table}[h]
\centering
\begin{tabular}{|c|c|c|c|}%{|c|p{2cm}|p{2cm}|p{2cm}|}
  % after \\: \hline or \cline{col1-col2} \cline{col3-col4} ...
  \hline
  $q_{\pi_k}(s,a)$ & \blue{$a_\ell$} & \blue{$a_0$} & \blue{$a_r$} \\
  \hline
  \blue{$s_1$} & $-1+\gamma v_{\pi_k}(s_1)$ & $0+\gamma v_{\pi_k}(s_1)$ & $1+\gamma v_{\pi_k}(s_2)$\\
  \hline
  \blue{$s_2$} & $0+\gamma v_{\pi_k}(s_1)$ & $1+\gamma v_{\pi_k}(s_2)$ & $-1+\gamma v_{\pi_k}(s_2)$\\
  \hline
\end{tabular}
%\caption{The expression of $q_{\pi_k}(s,a)$ for the example in Figure~\ref{figChapterVIPI_demoPI}.}
%\label{table_qvalueExpressionForDemoPI}
\end{table}
}

\pause
Substituting $v_{\pi_0}(s_1)=-10, v_{\pi_0}(s_2)=-9$ and $\gamma=0.9$ gives
\begin{table}[h!]
\centering
\resizebox{0.4\columnwidth}{!}{
\begin{tabular}{|c|c|c|c|}%{|c|p{2cm}|p{2cm}|p{2cm}|}
  % after \\: \hline or \cline{col1-col2} \cline{col3-col4} ...
  \hline
  $q_{\pi_0}(s,a)$ & \blue{$a_\ell$} & \blue{$a_0$} & \blue{$a_r$} \\
  \hline
  \blue{$s_1$} & $-10$ & $-9$ & $\textcolor{red}{-7.1}$\\
  \hline
  \blue{$s_2$} & $-9$ & $\textcolor{red}{-7.1}$ & $-9.1$\\
  \hline
\end{tabular}
}
%\caption{The value of $q_{\pi_k}(s,a)$ when $k=0$.}
%\label{table_qvalueExpressionForDemoPIk0}
\end{table}

\pause
By seeking the greatest value of $q_{\pi_0}$, the improved new policy is:
$$\pi_1(a_r|s_1)=1, \quad \pi_1(a_0|s_2)=1.$$
This policy is optimal after one iteration! In your programming, should continue until the stopping criterion is satisfied.
\end{frame}

%--------------------------------------
\begin{frame}
\frametitle{Policy iteration algorithm - Simple example}

\textbf{Exercise:} Set the left cell as the target area.

\vspace{10pt}
\pause
\begin{itemize}
\item Now we have another powerful algorithm searching for optimal policies!
\item Next let's apply it and see what we can find.
\end{itemize}

\end{frame}
%--------------------------------------
\begin{frame}
\frametitle{Policy iteration algorithm - Complicated example}
\hl Setting: $r_{\rm boundary}=-1$, $r_{\rm forbidden}=-10$, $r_{\rm target}=1$, $\gamma=0.9$.

\hl Let's check out the intermediate policies and state values.
\vspace{-10pt}
\captionsetup[subfigure]{labelformat=empty}
\begin{figure}[]
  \centering
  \visible<1->{
  \subfloat[$\pi_0$ and $v_{\pi_0}$]{
  \includegraphics[width=0.23\linewidth]{fig_gridPolicy_PIPolicy0.pdf}
  \includegraphics[width=0.23\linewidth]{fig_gridPolicy_PIValue0.pdf}
  }}
  \visible<2->{\subfloat[$\pi_1$ and $v_{\pi_1}$]{
  \includegraphics[width=0.23\linewidth]{fig_gridPolicy_PIPolicy1.pdf}
  \includegraphics[width=0.23\linewidth]{fig_gridPolicy_PIValue1.pdf}
  }}\\
  \visible<3->{\subfloat[$\pi_2$ and $v_{\pi_2}$]{
  \includegraphics[width=0.23\linewidth]{fig_gridPolicy_PIPolicy2.pdf}
  \includegraphics[width=0.23\linewidth]{fig_gridPolicy_PIValue2.pdf}
  }}
  \visible<4->{\subfloat[$\pi_3$ and $v_{\pi_3}$]{
  \includegraphics[width=0.23\linewidth]{fig_gridPolicy_PIPolicy3.pdf}
  \includegraphics[width=0.23\linewidth]{fig_gridPolicy_PIValue3.pdf}
  }}
\end{figure}
\end{frame}
%--------------------------------------
\begin{frame}
\frametitle{Policy iteration algorithm - Complicated example}
\hl Interesting pattern of the policies and state values
\captionsetup[subfigure]{labelformat=empty}
\vspace{-10pt}
\begin{figure}[]
  \centering
  \subfloat[$\pi_4$ and $v_{\pi_4}$]{
  \includegraphics[width=0.23\linewidth]{fig_gridPolicy_PIPolicy4.pdf}
  \includegraphics[width=0.23\linewidth]{fig_gridPolicy_PIValue4.pdf}
  }
  \visible<2->{\subfloat[$\pi_5$ and $v_{\pi_5}$]{
  \includegraphics[width=0.23\linewidth]{fig_gridPolicy_PIPolicy5.pdf}
  \includegraphics[width=0.23\linewidth]{fig_gridPolicy_PIValue5.pdf}
  }}\\
  \visible<3->{\vspace{-10pt}
  \begin{center}$\vdots$\qquad\qquad\qquad\qquad\qquad\qquad\qquad\qquad$\vdots$\end{center}
  \vspace{-10pt}}
  \visible<4->{\subfloat[$\pi_{9}$ and $v_{\pi_{9}}$]{
  \includegraphics[width=0.23\linewidth]{fig_gridPolicy_PIPolicy9.pdf}
  \includegraphics[width=0.23\linewidth]{fig_gridPolicy_PIValue9.pdf}
  }}
  \visible<5->{\subfloat[$\pi_{10}$ and $v_{\pi_{10}}$]{
  \includegraphics[width=0.23\linewidth]{fig_gridPolicy_PIPolicy10.pdf}
  \includegraphics[width=0.23\linewidth]{fig_gridPolicy_PIValue10.pdf}
  }}
\end{figure}
\end{frame}
%--------------------------------------
\AtBeginSection[]% put it to the start of each section
{
  \begin{frame}
    \frametitle{Outline}
    \tableofcontents[currentsection]
  \end{frame}
}
\section{Truncated policy iteration algorithm}
\begin{frame}
\frametitle{Compare value iteration and policy iteration}
\textbf{Policy iteration:} \red{start from $\pi_0$}
\begin{itemize}
\item Policy evaluation (PE):
$$v_{\pi_k}=r_{\pi_k}+\gamma  P_{\pi_k} v_{\pi_k}$$
\item Policy improvement (PI):
\begin{align*}
\pi_{k+1}=\arg\max_{\pi} (r_\pi +\gamma  P_\pi \textcolor{orange}{v_{\pi_k}})
\end{align*}
\end{itemize}
\pause
\textbf{Value iteration:} \red{start from $v_0$}
\begin{itemize}
\item Policy update (PU): $$\pi_{k+1}=\arg\max_\pi (r_\pi +\gamma  P_\pi \textcolor{orange}{v_k})$$
\item Value update (VU): $$v_{k+1}=r_{\pi_{k+1}} +\gamma  P_{\pi_{k+1}} v_k$$
\end{itemize}

\end{frame}
%--------------------------------------
\begin{frame}
\frametitle{Compare value iteration and policy iteration}
\hl The two algorithms are very similar:
\begin{align*}
\text{Policy iteration: } \pi_0\xrightarrow{PE} &v_{\pi_0}\xrightarrow{PI} \pi_1 \xrightarrow{PE} v_{\pi_1}\xrightarrow{PI} \pi_2 \xrightarrow{PE} v_{\pi_2}\xrightarrow{PI} \dots \\
\text{Value iteration: \ \quad\qquad} &\,u_{0}\xrightarrow{PU} \pi_1' \xrightarrow{VU} u_{1}\xrightarrow{PU} \pi_2' \xrightarrow{VU} u_{2}\xrightarrow{PU} \dots
\end{align*}
\begin{itemize}
\item PE=policy evaluation. PI=policy improvement.
\item PU=policy update. VU=value update.
\end{itemize}
\end{frame}
%--------------------------------------
\begin{frame}
\frametitle{Compare value iteration and policy iteration}

\hl Let's compare the steps carefully:
\vspace{-10pt}
\begin{table}[]
\centering
{
\resizebox{\columnwidth}{!}{
    \begin{tabular}{l|l|l|p{3cm}}
              \hline
              % after \\: \hline or \cline{col1-col2} \cline{col3-col4} ...
            &  Policy iteration algorithm & Value iteration algorithm & Comments\\
              \hline
            1) Policy: &  $\pi_0$ & N/A & \\
              \hline
            2) Value: &  $v_{\pi_0}=r_{\pi_0}+\gamma P_{\pi_0}v_{\pi_0}$ &   \blue{$v_0\doteq v_{\pi_0}$} & \\
              \hline
            3) Policy: &  $\pi_{1}=\arg\max_{\pi} (r_\pi +\gamma  P_\pi v_{\pi_0})$ &  $\pi_{1}=\arg\max_{\pi} (r_\pi +\gamma  P_\pi v_0)$ & The two policies are the same\\
              \hline
            4) Value: &   \red{$v_{\pi_1}=r_{\pi_1}+\gamma P_{\pi_1}v_{\pi_1}$} &  \red{$v_1=r_{\pi_1}+\gamma P_{\pi_1}v_0$} & $v_{\pi_1}\ge v_1$ since $v_{\pi_1}\ge v_{\pi_0}$ \\
              \hline
            5) Policy: & $\pi_{2}=\arg\max_{\pi} (r_\pi +\gamma  P_\pi v_{\pi_1})$ &  $\pi_{2}'=\arg\max_{\pi} (r_\pi +\gamma  P_\pi v_1)$ & \\
            \hline
            \vdots & \vdots & \vdots & \vdots\\
            \hline
    \end{tabular}
}
}
\end{table}
\pause
\vspace{-5pt}
Remarks:
\begin{itemize}
\item They start from the same initial condition.
\item The first three steps are the same.
\item The fourth step becomes different:
\begin{itemize}
\item[-] In policy iteration, solving $v_{\pi_1}=r_{\pi_1}+\gamma P_{\pi_1}v_{\pi_1}$ requires an iterative algorithm (an infinite number of iterations)
\item[-] In value iteration, $v_1=r_{\pi_1}+\gamma P_{\pi_1}v_0$ is a one-step iteration
\end{itemize}
\end{itemize}
\end{frame}
%--------------------------------------
\begin{frame}
\frametitle{Compare value iteration and policy iteration}
More specifically, examine the step of \blue{solving $v_{\pi_1}=r_{\pi_1}+\gamma P_{\pi_1}v_{\pi_1}$}:
{\small
\begin{align*}
&\red{v_{\pi_1}^{(0)}=v_{0}}\\
\visible<2->{\text{value iteration}\leftarrow\red{v_1\longleftarrow}}
 &v_{\pi_1}^{(1)}=r_{\pi_1}+\gamma P_{\pi_1}v_{\pi_1}^{(0)}\\
&v_{\pi_1}^{(2)}=r_{\pi_1}+\gamma P_{\pi_1}v_{\pi_1}^{(1)}\\
&\quad\vdots  \\
\visible<4->{\text{\red{truncated policy iteration}}\leftarrow\red{\bar{v}_1\longleftarrow}} &v_{\pi_1}^{(j)}=r_{\pi_1}+\gamma P_{\pi_1}v_{\pi_1}^{(j-1)}\\
&\quad\vdots  \\
\visible<3->{\text{policy iteration}\leftarrow\red{v_{\pi_1}\longleftarrow}} &v_{\pi_1}^{(\infty)}=r_{\pi_1}+\gamma P_{\pi_1}v_{\pi_1}^{(\infty)}\\
\end{align*}
}
\vspace{-30pt}
\visible<5->{
\begin{itemize}
\item The \red{value iteration} algorithm computes \blue{once}.
\item The \red{policy iteration} algorithm computes \blue{an infinite number of iterations}.
\item The \red{truncated policy iteration} algorithm computes \blue{a finite number of iterations} (say $j$). The rest iterations from $j$ to $\infty$ are truncated.
\end{itemize}
}
\end{frame}
%--------------------------------------
\begin{frame}
\frametitle{Truncated policy iteration - Pseudocode}

\begin{figure}[t]
{\scriptsize
\begin{mdframed}[style=myAlgo,nobreak=true,frametitle={Pseudocode: Truncated policy iteration algorithm}]
{\fontfamily{cmss}\selectfont
\textbf{Initialization:} The probability model $p(r|s,a)$ and $p(s'|s,a)$ for all $(s,a)$ are known. Initial guess $\pi_0$.

\textbf{Aim:} Search for the optimal state value and an optimal policy.

While $v_k$ has not converged, for the $k$th iteration, do

\qquad \blue{Policy evaluation:}

\setlength{\leftskip}{2em}
Initialization: select the initial guess as $v_{k}^{(0)}=v_{k-1}$. The maximum iteration is set to be $j_{\rm truncate}$.

While $j< j_{\rm truncate}$, do

\setlength{\leftskip}{4em}  For every state $s\in\S$, do

\setlength{\leftskip}{6em}  $v_{k}^{(j+1)}(s)=\sum_{a}{\pi_k(a|s)}\left[\sum_{r}p(r|s,a)r + \gamma \sum_{s'}p(s'|s,a)v_{k}^{(j)}(s')\right]$

\setlength{\leftskip}{2em}
Set $v_{k}=v_{k}^{(j_{\rm truncate})}$

\setlength{\leftskip}{2em}
\blue{Policy improvement:}

\setlength{\leftskip}{2em}
For every state $s\in\S$, do

\setlength{\leftskip}{4em}
For every action $a\in\A(s)$, do

\setlength{\leftskip}{6em}
$q_{k}(s,a)=\sum_{r}p(r|s,a)r + \gamma \sum_{s'}p(s'|s,a)v_{k}(s')$

\setlength{\leftskip}{4em}
$a^*_k(s)=\arg\max_a q_{k}(s,a)$

\qquad\qquad
$\pi_{k+1}(a|s)=1$ if $a=a^*_k$, and $\pi_{k+1}(a|s)=0$ otherwise
}
\end{mdframed}
}
\end{figure}
\end{frame}
%--------------------------------------
\begin{frame}
\frametitle{Truncated policy iteration - Convergence}

\hl Will the truncation undermine convergence?

\begin{proposition}[Value Improvement]\label{proposition_TPIValueImprovement}
Consider the iterative algorithm for solving the policy evaluation step:
\begin{align*}
v_{\pi_k}^{(j+1)}=r_{\pi_k}+\gamma  P_{\pi_k} v_{\pi_k}^{(j)},\quad j=0,1,2,...
\end{align*}
If the initial guess is selected as $v_{\pi_k}^{(0)}=v_{\pi_{k-1}}$, it holds that
$$v_{\pi_k}^{(j+1)}\ge v_{\pi_k}^{(j)}$$
for every $j=0,1,2,\dots$.
\end{proposition}
For the proof, see the book.
\end{frame}
%--------------------------------------
\begin{frame}
\frametitle{Truncated policy iteration - Convergence}
\begin{figure}[h]
  \centering
  \includegraphics[width=0.5\linewidth]{fig_demoConvergenceThreeCurves}
  \caption{Illustration of the relationship among value iteration, policy iteration, and truncated policy iteration.}
  \label{figChapterVIPI_demoThreeAlgoConvergenceSpeed}
\end{figure}

\vspace{-10pt}

The convergence proof of PI is based on that of VI. Since VI converges, we know PI converges.
\end{frame}
%%--------------------------------------
%\begin{frame}
%\frametitle{Truncated policy iteration - Example}
%\hl Setup: The same as the previous example. Below is the initial policy.
%
%\begin{figure}[t]
%  \centering
%  \includegraphics[width=0.4\linewidth]{fig_gridPolicy_PIPolicy0.pdf}
%\end{figure}
%
%\hl Define $\|v_k-v^*\|$ as the state value error at time $k$. The stop criterion is $\|v_k-v^*\|<0.01$.
%
%\end{frame}%
%%--------------------------------------
%\begin{frame}
%\frametitle{Truncated policy iteration - Example}
%\begin{figure}[t]
%  \centering
%    \includegraphics[width=0.7\linewidth]{fig_compareVIPITPI}
%\end{figure}
%
%\hl ``Truncated policy iteration-$x$'' where $x=1,3,6,100$ refers to a truncated policy iteration algorithm where the policy evaluation step runs $x$ iterations.
%\end{frame}
%%--------------------------------------
%\begin{frame}
%\frametitle{Truncated policy iteration - Example}
%\begin{figure}[t]
%  \centering
%    \includegraphics[width=0.7\linewidth]{fig_compareVIPITPI}
%\end{figure}
%
%\hl The greater the value of $x$ is, the faster the value estimate converges.
%
%\hl However, the benefit of increasing $x$ drops quickly when $x$ is large.
%
%\hl In practice, run a few number of iterations in the policy evaluation step.
%
%\end{frame}
%--------------------------------------
\begin{frame}
\frametitle{Summary}

\hl Value iteration: it is the iterative algorithm solving the Bellman optimality equation: given an initial value $v_0$,
\begin{align*}
\left.
  \begin{array}{c}
    v_{k+1}=\max_{\pi}(r_\pi+\gamma P_{\pi} v_k) \\
    \Updownarrow \\
    \left\{
                                                               \begin{array}{l}
                                                                 \text{Policy update: } \pi_{k+1}=\arg\max_{\pi}(r_\pi+\gamma P_\pi v_k) \\
                                                                 \text{Value update: }
                                                                 v_{k+1}= r_{\pi_{k+1}}+\gamma P_{\pi_{k+1}} v_k\\
                                                               \end{array}
                                                             \right. \\
  \end{array}
\right.
\end{align*}
\hl Policy iteration: given an initial policy $\pi_0$,
\begin{align*}
\left\{
                                                               \begin{array}{l}
                                                                 \text{Policy evaluation: } v_{\pi_k}=r_{\pi_k}+\gamma  P_{\pi_k} v_{\pi_k} \\
                                                                 \text{Policy improvement: }\pi_{k+1}=\arg\max_{\pi} (r_\pi +\gamma  P_\pi v_{\pi_k})\\
                                                               \end{array}
                                                             \right.
\end{align*}

\hl Truncated policy iteration
\end{frame}
%%%%%%%%%%%%%%%%%%%%%%%%%%%%%%%%%%%%%%%%%%%%%%%%%%%%%%%%%%%%%%%%%%%%%%%%%%%%%%%%%%%%%%%%%%%%%%%%%%
%\bibliographystyle{plainnat}
%\bibliography{myOwnPub,zsyReferenceAll}
\end{document}
