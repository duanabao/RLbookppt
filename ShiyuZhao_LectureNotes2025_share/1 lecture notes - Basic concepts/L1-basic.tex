\documentclass[xcolor=dvipsnames,mathserif,9pt]{beamer} %handout
%\usefonttheme{serif}%{structurebold}%{structuresmallcapsserif}%{serif}

%\input{before_document}
\usepackage{graphicx}
\usepackage{amsmath}
\usepackage{amssymb}
\usepackage[font=footnotesize]{caption} % set the captain font size to 8 (i.e. footnotesize)
\usepackage{subfig} % uses subfloats within a single float MUST after the package {caption}!!
\usepackage{natbib}
%\usepackage{cite} % sort the reference in the article by number or alphabatic
\usepackage{color}
\usepackage{algorithm} % options: boxed [section]
\usepackage{algpseudocode} % for algorithm
%\usepackage{enumerate}
\usepackage{enumitem} % directly use itemize, easily specify indent and everything
\setlist[itemize]{leftmargin=*,label=$\bullet$}%leftmargin=*,itemsep=0pt} %topsep=5pt
\setlist[enumerate]{label={\arabic*)}}
\usepackage{hyperref}
\usepackage{wrapfig}
\usepackage{textpos}
\usepackage{bibentry} % for publication list
\makeatletter\let\saved@bibitem\@bibitem\makeatother % make hyperref and bibentry compatible!!!
\nobibliography*
\usepackage{fancybox}% shadow for image
%\usepackage{empheq} % emphasize equations
\usepackage{bm}
\usepackage{arydshln} % for dashline in table or matrix
\linespread{1.3}
\usepackage{multimedia}

\usepackage{setspace} \setstretch{1.2}

\usepackage{framed}
\colorlet{shadecolor}{black!5}
% for box, page breakable, very good!!
\usepackage[framemethod=TikZ]{mdframed}%
\mdfdefinestyle{myFrame}{%
    linecolor=gray!15!white,%gray
    outerlinewidth=0.1pt,
    roundcorner=3pt,
    skipabove=15pt, % the space before the entire box
    skipbelow=15pt, % the space after the entire box. Please see the figure 2 in the manual, very clear!
    innertopmargin=10pt,%\baselineskip,
    innerbottommargin=10pt,%\baselineskip,
    %innerrightmargin=10pt,
    %innerleftmargin=10pt,
    splittopskip=\baselineskip,
    splitbottomskip=\baselineskip,
    backgroundcolor=gray!10!white,
    frametitlerule=true,
    frametitlebackgroundcolor=gray!20!white,
    frametitleaboveskip=5pt,
    frametitlebelowskip=5pt,
}
\mdfdefinestyle{myAlgo}{%
    linecolor=gray!100!white,%gray
    outerlinewidth=0.1pt,
    roundcorner=3pt,
    skipabove=15pt, % the space before the entire box
    skipbelow=15pt, % the space after the entire box. Please see the figure 2 in the manual, very clear!
    innertopmargin=10pt,%\baselineskip,
    innerbottommargin=10pt,%\baselineskip,
    %innerrightmargin=10pt,
    %innerleftmargin=10pt,
    splittopskip=\baselineskip,
    splitbottomskip=\baselineskip,
    backgroundcolor=gray!0!white,
    frametitlerule=true,
    frametitlebackgroundcolor=gray!20!white,
    frametitleaboveskip=5pt,
    frametitlebelowskip=5pt,
}


\usepackage{tikz}
\usetikzlibrary{calc} % for calculation functions in Tikz let, in commands in Tikz
\usetikzlibrary{shapes} % for block diagram
\usetikzlibrary{chains}
\usetikzlibrary{fit}
\usetikzlibrary{arrows}
\usetikzlibrary{decorations.text} % text along path

\newcommand{\blue}[1]{\textcolor{blue}{#1}}
\definecolor{myred}{RGB}{200,0,0}
\newcommand{\red}[1]{\textcolor{myred}{#1}} %magenta purple
\newcommand{\I}{\mathcal{I}}
\newcommand{\tr}{\mathrm{tr}}
\newcommand{\Null}{\mathrm{Null}}
\newcommand{\Range}{\mathrm{Range}}
\newcommand{\one}{\mathbf{1}}
\newcommand{\rank}{\mathrm{rank}}
\newcommand{\myspan}{\mathrm{span}}
\newcommand{\mydiag}{\mathrm{diag}}
\newcommand{\D}{\mathrm{d}}
\renewcommand{\d}{\mathrm{d}}
\newcommand{\blkdiag}{\mathrm{blkdiag}}
\newcommand{\sgn}{\mathrm{sgn}}
\newcommand{\T}{\mathrm{T}}
\newcommand{\myqed}{\hfill$\blacksquare$}
\newcommand{\ep}{\varepsilon}
\newcommand{\sig}{\mathrm{sig}_a}
%\newcommand{\sigep_}[1]{\sig(\ep_{#1})}
\newcommand{\R}{\mathbb{R}}
\newcommand{\A}{\mathcal{A}}
\newcommand{\G}{\mathcal{G}}
\newcommand{\E}{\mathbb{E}}
\newcommand{\X}{\mathcal{X}}
\newcommand{\V}{\mathcal{V}}
\newcommand{\N}{\mathcal{N}}
\newcommand{\M}{\mathcal{M}}
\renewcommand{\H}{\mathcal{H}}
\renewcommand{\L}{\mathcal{B}}
\renewcommand{\S}{\mathcal{S}}
\newcommand{\xe}{x_{\text{e}}}
%\newcommand{\Null}[1]{\mathrm{Null}\left(#1\right)}
\newcommand{\sk}[1]{\left[#1\right]_\times} % skew symmetric operator
\newcommand{\dia}[1]{\mathrm{diag}\left(#1\right)} % block diagnal matrix
%\renewcommand{\span}[1]{\mathrm{span}\left\{#1\right\}} % ERROR when redefine \span
\newcommand{\Var}{\mathrm{Var}}
\newcommand{\var}{\mathrm{var}}


\graphicspath{{figures/}}

% for tikz, theorem, lemma ... environments have already been declared. You don't need to declare, or you need to use other names than theorem or lemma, such as my_theorem.
%\newtheorem{my_theorem}{Theorem}
%\newtheorem{my_lemma}{Lemma}
\newtheorem{assumption}{Assumption} % necessary for beamer
%\newtheorem{my_remark}{Remark}
\newtheorem{proposition}{Proposition} % necessary for beamer
%\newtheorem{my_corollary}{Corollary}
%\newtheorem{my_example}{Example}
%\newtheorem{my_definition}{Definition}
%\newtheorem{my_problem}{Problem}

%##################################################
\newcommand{\pagetitle}[1]{\textbf{\textcolor{BlueViolet}{$\circ$ #1}}} %!!!
\newcommand{\pagehighlight}[1]{\textbf{\textcolor{Brown}{#1}}} %!!!
\newcommand{\mypause}{\pause} % this is useful for slide show. if you don't want pause any more, just set it as blank
%\newcommand{\mybullet}{\textcolor{BlueViolet}{$\blacksquare$} }%{$\rhd$ }
%\newcommand{\myhighsign}{$\star$ }% the sign to highlight a sentence

%##################################################
% To highlight equation. Example: \begin{align*} \boxed{xxx} \end{align*}
% does not support multiline equations
% put color to \boxed math command
\newcommand*{\boxcolor}{gray}
\makeatletter
\renewcommand{\boxed}[1]{\textcolor{\boxcolor}{%
%\tikz[baseline={([yshift=-1ex]current bounding box.center)}] \node [rectangle, minimum width=1ex,rounded corners,draw] {\normalcolor\m@th$\displaystyle#1$};}}
\tikz[baseline={([yshift=-1ex]current bounding box.center)}] \node [rectangle, minimum width=2ex,rounded corners,draw] {\normalcolor\m@th$\displaystyle#1$};}}
\makeatother

%##################################################
% set my own theme
\def\structureHeight{9mm}
\usetheme[height=\structureHeight]{Rochester}
\usecolortheme[RGB={0,0,128}]{structure}
\setbeamertemplate{items}[circle]%rectangle, triangle,circle
\setbeamertemplate{blocks}[rounded][shadow=true]
\setbeamertemplate{navigation symbols}{}
%\addtobeamertemplate{frametitle}{} % specify the logo
%{
%    \begin{textblock*}{100mm}(.87\textwidth,-\structureHeight)
%        \includegraphics[height=6.6mm,width=3cm,keepaspectratio]{../common_figures_private/westlake_logo.png} % add logo
%    \end{textblock*}
%}
\addtobeamertemplate{frametitle}{\vskip4pt}{} % specify
%\setbeamerfont{frametitle}{size=\large}
\definecolor{mylightgray}{RGB}{240 240 240}
\definecolor{mykhaki}{RGB}{240 230 140}% khaki color
\definecolor{mylightYellow}{RGB}{255,255,224} % light yellow
%\setbeamercolor{beamercolor1}{bg=mylightgray, fg=black}
%\setbeamercolor{beamercolor2}{bg=mylightYellow,fg=black}%{bg=yellow!90!white, fg=black}
% background and foreground color
\setbeamercolor{background canvas}{bg=black!0!white} % background color of every slide! My previous value was black!10!white for my lecture videos!
\setbeamercolor{normal text}{bg=black!10!white} % background color for e.g. theorem environment. When canvas is 10, here it can be 20; the bg for normal text changes the color of hidden text when you use overlay

\setbeamercolor{block title}{bg=mykhaki,fg=black}
\defbeamertemplate{footline}{zsy_frameNumber}
{%
  \hspace{5pt}  \emph{Shiyu Zhao}
  \hspace*{\fill}%
  \usebeamercolor[fg]{page number in head/foot}%
  \insertframenumber\,/\,\inserttotalframenumber \vspace{0pt} \hspace{5pt}
  \vskip5pt
}
\setbeamertemplate{footline}[zsy_frameNumber]
%##################################################
\setbeamercovered{transparent=0} % a good value for transparent text is 20
% when using the overlay commands like \onslide or \uncover, the text will NOT be invisible, instead it will be like transparent
%\pause will also have the transparent effect: command \pause is easy to use: to make it invisible, change the value to zero.




\usepackage{multimedia}
\linespread{1.1}
\newcommand{\hl}{$\triangleright$ }
%\newcommand{\blue}[1]{\textcolor{blue}{#1}} % skew symmetric operator
\begin{document}

%%%%%%%%%%%%%%%%%%%%%%%%%%%%%%%%%%%%%%%%%%%%%%%%%%%%%%%%%%%%%%%%%%%%%%%%%%%%%%%%%
% define the author, date etc. information
%\subtitle{Mathematical and Biological Foundation for Reinforcement Learning}
\title{Lecture 2: State Value and Bellman Equation}

\author{Shiyu Zhao
\newline
\newline {\small Department of Artificial Intelligence}
\newline {\small Westlake University}
}
%\logo{\includegraphics[width=1cm,height=1cm,keepaspectratio]{NUSLogo.png}~}
%\date{\today}
\date{}
\subject{}


%%%%%%%%%%%%%%%%%%%%%%%%%%%%%%%%%%%%%%%%%%%%%%%%%%%%%%%%%%%%%%%%%%%%%%%%%%%%%%%%%

{
\setbeamertemplate{footline}{} % remove the frame number of the title page
\begin{frame}
    %\frametitle{Lecture: Networked Dynamic Systems}
    \addtocounter{framenumber}{-1} % discounter the title page, otherwise the frame number starts from 2 instead of 1
    \titlepage % this only gives the author etc. information
\end{frame}
}


%------------------------------------------
\begin{frame}
\frametitle{Outline}
\begin{figure}[h]
  \centering
\includegraphics[width=0.8\linewidth]{fig_bookMap1.pdf}
\end{figure}
\end{frame}
%------------------------------------------
\begin{frame}
\frametitle{Contents}

\begin{itemize}
\item First, introduce fundamental concepts in reinforcement learning (RL) by examples.
\vspace{10pt}
\item Second, formalize the concepts in the context of Markov decision processes.
\end{itemize}
\end{frame}
%%%%%%%%%%%%%%%%%%%%%%%%%%%%%%%%%%%%%%%%%%%%%%%%%%%%%%%%%%%%%%%%%%%%%%%%%%%%%%%%%%%%%%%%%%%%%%%%%%


\begin{frame}
\frametitle{A grid-world example}

\begin{figure}[h]
  \centering
  \includegraphics[width=0.3\linewidth]{fig_gridIntuition}\quad
  \visible<1->{\includegraphics[width=0.31\linewidth]{fig_3x3gridFog.pdf}}
\end{figure}

An illustrative example used throughout this course:
\begin{itemize}
\item Grid of cells: Accessible/forbidden/target cells, boundary.
\item Very easy to understand and useful for illustration
\end{itemize}

\pause
Task:
\begin{itemize}
\item Given any starting area, find a ``good'' way to the target.
\item How to define ``good''? Avoid forbidden cells, detours, or boundary.
\end{itemize}

\end{frame}
%******************************************************************************
\begin{frame}
\frametitle{State}
\blue{\emph{State}}: The status of the agent with respect to the environment.
\begin{itemize}
\item For the grid-world example, the location of the agent is the state. There are nine possible locations and hence nine states: $s_1,s_2,\dots,s_9$.
\end{itemize}

\begin{figure}[h]
  \centering
  \includegraphics[width=0.3\linewidth]{fig_gridDemoState}
\end{figure}
\pause
\blue{\emph{State space:}} the set of all states $\S=\{s_i\}_{i=1}^9$.

\end{frame}
%******************************************************************************
\begin{frame}
\frametitle{Action}

\emph{\blue{Action}}: For each state, there are five possible actions: $a_1,\dots,a_5$
\begin{itemize}
\item $a_1$: move upward;
\item $a_2$: move rightward;
\item $a_3$: move downward;
\item $a_4$: move leftward;
\item $a_5$: stay still;
\end{itemize}
\begin{figure}[h]
  \centering
  \includegraphics[width=0.3\linewidth]{fig_gridDemoAction} \qquad \includegraphics[width=0.25\linewidth]{fig_gridDemoState}
\end{figure}
\pause
\blue{Action space of a state}: the set of all possible actions of a state. $\A(s_i)=\{a_k\}_{k=1}^5$.

\pause
\blue{Question}: can different states have different sets of actions?
\end{frame}
%******************************************************************************
\begin{frame}
\frametitle{State transition}
\begin{figure}
  \centering
  %\vspace{-30pt}
  \includegraphics[width=0.2\linewidth]{fig_gridDemoState}
  %\vspace{10pt}
\end{figure}
When taking an action, the agent may move from one state to another. Such a process is called \emph{\blue{state transition}}.
\begin{itemize}
\pause
\item Example: In state $s_1$, if we choose action $a_2$, then what is the next state?
$$s_1\xrightarrow{a_2}s_2$$

\pause
\item Example: In state $s_1$, if we choose action $a_1$, then what is the next state?
$$s_1\xrightarrow{a_1}s_1$$
\end{itemize}
\pause
\begin{itemize}
\item State transition describes the interaction with the environment.
\pause
\item \blue{Question:} Can we define the state transition in other ways? Simulation vs physics
\end{itemize}
\end{frame}
%******************************************************************************
\begin{frame}
\frametitle{State transition}
\begin{figure}
  \centering
  %\vspace{-30pt}
  \includegraphics[width=0.2\linewidth]{fig_gridDemoState}
  %\vspace{10pt}
\end{figure}
Pay attention to \blue{forbidden areas}: Example: in state $s_5$, if we choose action $a_2$, then what is the next state?
\begin{itemize}
\pause
\item Case 1: the forbidden area is accessible but with penalty. Then,
$$s_5\xrightarrow{a_2}s_6$$

\pause
\item Case 2: the forbidden area is inaccessible (e.g., surrounded by a wall)
$$s_5\xrightarrow{a_2}s_5$$
\end{itemize}
\pause
\begin{itemize}
\item[-] The lectures consider the first case, which is more general and challenging.
\item[-] The assignments consider the second case!
\end{itemize}
\end{frame}
%******************************************************************************
\begin{frame}
\frametitle{State transition}
\begin{figure}
  \centering
  %\vspace{-30pt}
  \includegraphics[width=0.2\linewidth]{fig_gridDemoState}
  %\vspace{10pt}
\end{figure}
Tabular representation: We can use a table to describe the state transition:
%\vspace{-10pt}
\pause
\begin{table}[h!]
\centering
\resizebox{\columnwidth}{!}{
\begin{tabular}{c|c|c|c|c|c}
  % after \\: \hline or \cline{col1-col2} \cline{col3-col4} ...
  \hline
   & \blue{$a_1$ (upward)} & \blue{$a_2$ (rightward)} & \blue{$a_3$ (downward)} & \blue{$a_4$ (leftward)} & \blue{$a_5$ (still)} \\
  \hline
  \blue{$s_1$} & $s_1$ & $s_2$ & $s_4$ & $s_1$ & $s_1$\\
  \hline
  \blue{$s_2$} & $s_2$ & $s_3$ & $s_5$ & $s_1$ & $s_2$\\
  \hline
  \blue{$s_3$} & $s_3$ & $s_3$ & $s_6$ & $s_2$ & $s_3$\\
  \hline
  \blue{$s_4$} & $s_1$ & $s_5$ & $s_7$ & $s_4$ & $s_4$\\
  \hline
  \blue{$s_5$} & $s_2$ & $s_6$ & $s_8$ & $s_4$ & $s_5$\\
  \hline
  \blue{$s_6$} & $s_3$ & $s_6$ & $s_9$ & $s_5$ & $s_6$\\
  \hline
  \blue{$s_7$} & $s_4$ & $s_8$ & $s_7$ & $s_7$ & $s_7$\\
  \hline
  \blue{$s_8$} & $s_5$ & $s_9$ & $s_8$ & $s_7$ & $s_8$\\
  \hline
  \blue{$s_9$} & $s_6$ & $s_9$ & $s_9$ & $s_8$ & $s_9$\\
  \hline
\end{tabular}
}
\end{table}
\pause
Can only represent \emph{deterministic} cases.
\end{frame}
%******************************************************************************
\begin{frame}
\frametitle{State transition}
\begin{figure}
  \centering
  %\vspace{-30pt}
  \includegraphics[width=0.25\linewidth]{fig_gridDemoState}
  %\vspace{10pt}
\end{figure}
\emph{State transition probability:} use probability to describe state transition!

\begin{itemize}
\pause
\item Intuition: In state $s_1$, if we choose action $a_2$, the next state is $s_2$.

\pause
\item Math:
\begin{align*}
p(s_2|s_1,a_2)&=1\\
p(s_i|s_1,a_2)&=0\quad \forall i\ne 2
\end{align*}
\end{itemize}

\pause
Here it is a \blue{deterministic} case. The state transition could be \blue{stochastic} (for example, wind gust).
\end{frame}
%******************************************************************************
\begin{frame}
\frametitle{Policy}
\emph{\blue{Policy}} tells the agent what actions to take in a state.

\visible<2->{
\textbf{Intuitive representation:} We use \emph{arrows} to describe a policy.

\begin{figure}[h]
  \centering
  \includegraphics[width=0.25\linewidth]{fig_gridDemoPolicy}
\end{figure}
}

\visible<3->{
Based on this policy, we get the following trajectories with different starting points.

\begin{figure}[h]
  \centering
  \includegraphics[width=0.25\linewidth]{fig_gridDemoPolicy_traj1}\quad
  \includegraphics[width=0.25\linewidth]{fig_gridDemoPolicy_traj2}\quad
  \includegraphics[width=0.25\linewidth]{fig_gridDemoPolicy_traj3}
\end{figure}}

\end{frame}
%******************************************************************************
\begin{frame}
\frametitle{Policy}
\begin{figure}[h]
  \centering
  \includegraphics[width=0.25\linewidth]{fig_gridDemoPolicy}
\end{figure}

\textbf{Mathematical representation:} using conditional probability

\pause
For example, for state $s_1$:
\begin{align*}
\pi(a_1|s_1)&=0\\
\pi(a_2|s_1)&=1\\
\pi(a_3|s_1)&=0\\
\pi(a_4|s_1)&=0\\
\pi(a_5|s_1)&=0
\end{align*}

\pause
It is a \textcolor{blue}{deterministic} policy.

\end{frame}
%******************************************************************************
\begin{frame}
\frametitle{Policy}

There are \blue{stochastic} policies.

\visible<2->{
For example:
\begin{figure}[h]
  \centering
  \includegraphics[width=0.3\linewidth]{fig_gridDemoPolicyStochastic}
\end{figure}
}

\visible<3->{
In this policy, for $s_1$:
\begin{align*}
\pi(a_1|s_1)&=0\\
\pi(a_2|s_1)&=0.5\\
\pi(a_3|s_1)&=0.5\\
\pi(a_4|s_1)&=0\\
\pi(a_5|s_1)&=0
\end{align*}
}
\end{frame}
%******************************************************************************
\begin{frame}
\frametitle{Policy}

\begin{figure}[h]
  \centering
  \includegraphics[width=0.25\linewidth]{fig_gridDemoPolicyStochastic}
\end{figure}

\textbf{Tabular representation} of a policy: how to use this table.
\vspace{-10pt}
\begin{table}[h!]
\centering
\resizebox{\columnwidth}{!}{
\begin{tabular}{c|c|c|c|c|c}
  % after \\: \hline or \cline{col1-col2} \cline{col3-col4} ...
  \hline
   & \blue{$a_1$ (upward)} & \blue{$a_2$ (rightward)} & \blue{$a_3$ (downward)} & \blue{$a_4$ (leftward )} & \blue{$a_5$ (still)} \\
  \hline
  \blue{$s_1$} & 0 & 0.5 & 0.5 & 0 & 0 \\
  \hline
  \blue{$s_2$} & 0 & 0 & 1 & 0 & 0 \\
  \hline
  \blue{$s_3$} & 0 & 0 & 0 & 1 & 0 \\
  \hline
  \blue{$s_4$} & 0 & 1 & 0 & 0 & 0 \\
  \hline
  \blue{$s_5$} & 0 & 0 & 1 & 0 & 0 \\
  \hline
  \blue{$s_6$} & 0 & 0 & 1 & 0 & 0 \\
  \hline
  \blue{$s_7$} & 0 & 1 & 0 & 0 & 0 \\
  \hline
  \blue{$s_8$} & 0 & 1 & 0 & 0 & 0 \\
  \hline
  \blue{$s_9$} & 0 & 0 & 0 & 0 & 1 \\
  \hline
\end{tabular}
}
\end{table}
Can represent either \emph{deterministic} or \emph{stochastic} cases.

\end{frame}
%******************************************************************************
\begin{frame}
\frametitle{Reward}

\textbf{Reward is one of the most unique concepts of RL.}

\emph{Reward}: a real number we get after taking an action.
\begin{itemize}
\item A \blue{positive} reward represents \blue{encouragement} to take such actions.
\item A \blue{negative} reward represents \blue{punishment} to take such actions.
\end{itemize}

\pause
\vspace{10pt}
Questions:
\begin{itemize}
\item Can positive indicate punishment and negative indicate encouragement?
\begin{itemize}
\item[-] Yes.
\item[-] In this case, reward may be called \emph{cost}.
\end{itemize}
\pause
\item What about a zero reward?
\begin{itemize}
\item[-] Relative values matter, not absolute values.
\item[-] Example: $r=\{+1,-1\}$ becomes $r=\{+2,0\}$ will not change the optimal policy. Details will be shown in the future.
\end{itemize}
\end{itemize}

\end{frame}
%******************************************************************************
\begin{frame}
\frametitle{Reward}
\vspace{-10pt}
\begin{figure}
  \centering
  %\vspace{-30pt}
  \includegraphics[width=0.25\linewidth]{fig_gridDemoState}
  %\vspace{10pt}
\end{figure}

In the grid-world example, the rewards are designed as follows:
\begin{itemize}
\item If the agent attempts to get out of the boundary, let $r_{\mathrm{bound}}=-1$
\item If the agent attempts to enter a forbidden cell, let $r_{\mathrm{forbid}}=-1$
\item If the agent reaches the target cell, let $r_{\mathrm{target}}=+1$
\item Otherwise, the agent gets a reward of $r=0$.
\end{itemize}

\pause

Reward can be interpreted as a \textbf{human-machine interface}, with which we can guide
the agent to behave as what we expect.
\begin{itemize}
\item For example, with the above designed rewards,
the agent will try to avoid getting out of the boundary or stepping into the forbidden
cells.
\end{itemize}
\end{frame}
%******************************************************************************
\begin{frame}
\frametitle{Reward}

\begin{figure}[h]
  \centering
  \includegraphics[width=0.23\linewidth]{fig_gridDemoState}
\end{figure}

\textbf{Tabular representation} of \emph{reward transition}: how to use the table?
\vspace{-10pt}
\begin{table}[h]
\centering
\resizebox{\columnwidth}{!}{
\begin{tabular}{c|c|c|c|c|c}
  % after \\: \hline or \cline{col1-col2} \cline{col3-col4} ...
  \hline
  & \blue{$a_1$ (upward)} & \blue{$a_2$ (rightward)} & \blue{$a_3$ (downward)} & \blue{$a_4$ (leftward )} & \blue{$a_5$ (still)} \\
  \hline
  \blue{$s_1$} & $r_{\mathrm{bound}}$ & 0 & 0 & $r_{\mathrm{bound}}$ & 0\\
  \hline
  \blue{$s_2$} & $r_{\mathrm{bound}}$ & 0 & 0 & 0 & 0\\
  \hline
  \blue{$s_3$} & $r_{\mathrm{bound}}$ & $r_{\mathrm{bound}}$ & $r_{\mathrm{forbid}}$ & 0 & 0\\
  \hline
  \blue{$s_4$} & 0 & 0 & $r_{\mathrm{forbid}}$ & $r_{\mathrm{bound}}$ & 0\\
  \hline
  \blue{$s_5$} & 0 & $r_{\mathrm{forbid}}$ & 0 & 0 & 0\\
  \hline
  \blue{$s_6$} & 0 & $r_{\mathrm{bound}}$ & $r_{\mathrm{target}}$ & 0 & $r_{\mathrm{forbid}}$\\
  \hline
  \blue{$s_7$} & 0 & 0 & $r_{\mathrm{bound}}$ & $r_{\mathrm{bound}}$ & $r_{\mathrm{forbid}}$\\
  \hline
  \blue{$s_8$} & 0 & $r_{\mathrm{target}}$ & $r_{\mathrm{bound}}$ & $r_{\mathrm{forbid}}$ & 0\\
  \hline
  \blue{$s_9$} & $ r_{\mathrm{forbid}}$ & $r_{\mathrm{bound}}$  & $r_{\mathrm{bound}}$ & 0 & $
  r_{\mathrm{target}}$\\
  \hline
\end{tabular}
}
\end{table}
Can only represent \emph{deterministic} cases.
\end{frame}
%******************************************************************************
\begin{frame}
\frametitle{Reward}
\begin{figure}
  \centering
  %\vspace{-30pt}
  \includegraphics[width=0.23\linewidth]{fig_gridDemoState}
  %\vspace{10pt}
\end{figure}
\textbf{Mathematical description}: conditional probability

\begin{itemize}
\item Intuition: In state $s_1$, if we choose action $a_1$, the reward is $-1$.

\item Math: $p(r=-1|s_1,a_1)=1$ and $p(r\ne-1|s_1,a_1)=0$
\end{itemize}

\pause
Remarks:
\begin{itemize}
\item Here it is a deterministic case. The reward transition could be stochastic.

For example, if you study hard, you will get rewards. But how much is uncertain.

%\item The reward depends on the state and action, but not the next state (for example, consider $s_1,a_1$ and $s_1,a_5$).
\end{itemize}
\end{frame}
%******************************************************************************
\begin{frame}
\frametitle{Trajectory and return}

\begin{figure}[h]
  \centering
  \includegraphics[width=0.3\linewidth]{fig_gridDemoReturnPath1}
\end{figure}

A \blue{\emph{trajectory}} is a state-action-reward chain:
\begin{align*}%\label{eq_finiteTrajectoryDemo}
s_1\xrightarrow[r=0]{a_2}s_2
\pause
\xrightarrow[r=0]{a_3} s_5
\pause
\xrightarrow[r=0]{a_3} s_8
\pause
\xrightarrow[r=1]{a_2} s_9
\end{align*}

\pause
The \blue{\emph{return}} of this trajectory is the sum of all the rewards collected along the trajectory:
\begin{align*}%\label{eq_finiteTrajectoryReturnDemo}
\mathrm{return}=0+0+0+1=1
\end{align*}

\end{frame}
%******************************************************************************
\begin{frame}
\frametitle{Trajectory and return}

\begin{figure}[h]
  \centering
  \includegraphics[width=0.3\linewidth]{fig_gridDemoReturnPath2}
\end{figure}

A different policy gives a different trajectory:
$$s_1\xrightarrow[r=0]{a_3}s_4 \xrightarrow[r=-1]{a_3} s_7 \xrightarrow[r=0]{a_2} s_8 \xrightarrow[r=+1]{a_2} s_9$$

\pause
The return of this path is:
\begin{align*}
\mathrm{return}=0-1+0+1=0
\end{align*}

\end{frame}
%******************************************************************************
\begin{frame}
\frametitle{Trajectory and return}

\begin{figure}[h]
  \centering
  \includegraphics[width=0.25\linewidth]{fig_gridDemoReturnPath1}\quad
  \includegraphics[width=0.25\linewidth]{fig_gridDemoReturnPath2}
\end{figure}

Which policy is better?
\begin{itemize}
\item \textbf{Intuition}: the first is better, because it avoids the forbidden areas.
\item \textbf{Mathematics}: the first one is better, since it has a greater return!

\pause
\item Return can be used to evaluate whether a policy is good or not (see details in the next lecture)!
\end{itemize}
%What is the ultimate goal of RL? Finding optimal policies! (see details in the next lectures)
\end{frame}
%******************************************************************************
\begin{frame}
\frametitle{Discounted return}
\vspace{-20pt}
\begin{figure}[h]
  \centering
  \includegraphics[width=0.35\linewidth]{fig_gridDemoReturnPath1Infinite}
\end{figure}

A trajectory may be infinitely long:
\begin{align*}
s_1\xrightarrow{a_2}s_2 \xrightarrow{a_3} s_5 \xrightarrow{a_3} s_8 \xrightarrow{a_2} s_9
\pause
\textcolor{blue}{\xrightarrow{a_5} s_9 \xrightarrow{a_5} s_9 \dots}
\end{align*}

\pause
The return is
\begin{align*}
\mathrm{return}
&=0+0+0+1\pause\blue{+1+1+\dots}\pause=\infty
\end{align*}
\pause
The definition is invalid since the return diverges!
\end{frame}
%******************************************************************************
\begin{frame}
\frametitle{Discounted return}

\begin{figure}[h]
  \centering
  \includegraphics[width=0.3\linewidth]{fig_gridDemoReturnPath1Infinite}
\end{figure}

Need to introduce a \blue{\emph{discount rate}} $\gamma\in(0,1)$:
\begin{align*}%\label{eq_infiniteTrajectoryReturnDemo}
\blue{\text{discounted return}}
&=0+\gamma0+\gamma^20+\gamma^31+\gamma^41+\gamma^51+\dots\\
&=\gamma^3(1+\gamma+\gamma^2+\dots)=\gamma^3\frac{1}{1-\gamma}.
\end{align*}

Roles: 1) the sum becomes finite; 2) balance the far and near future rewards:
\begin{itemize}
\item If $\gamma$ is close to 0, the value of the discounted return is dominated by the rewards obtained in the near future.
\item If $\gamma$ is close to 1, the value of the discounted return is dominated by the rewards obtained in the far future.
\end{itemize}
\end{frame}
%******************************************************************************
\begin{frame}
\frametitle{Episode}

When interacting with the environment following a policy, the agent may stop at some \emph{terminal states}. The resulting trajectory is called an \emph{episode} (or a trial).

\begin{figure}[h]
  \centering
  \includegraphics[width=0.3\linewidth]{fig_gridDemoReturnPath1}
\end{figure}

\pause
Example: episode
\begin{align*}%\label{eq_finiteTrajectoryDemo}
s_1\xrightarrow[r=0]{a_2}s_2 \xrightarrow[r=0]{a_3} s_5 \xrightarrow[r=0]{a_3} s_8 \xrightarrow[r=1]{a_2} s_9
\end{align*}

An episode is usually assumed to be a finite trajectory. Tasks with episodes are called \emph{episodic tasks}.

\end{frame}

%******************************************************************************
\begin{frame}
\frametitle{Episode}

Some tasks may have no terminal states, meaning the interaction with the environment will never end. Such tasks are called \emph{continuing tasks}.

\pause
\vspace{10pt}
In the grid-world example, should we stop after arriving the target?

\begin{itemize}
\item \blue{Episodic task + Terminal state}
\item \blue{Continuing task + Absorbing state:} Treat the target state as a special \emph{absorbing state}, entering which will lead to staying there forever.
\item \blue{Continuing task + Normal state:} We don't need to distinguish the target state from the others and treat it as a \emph{normal state}. The agent can still leave the target state and gain $r=+1$ when entering the target state.
    \begin{itemize}
    \item[-] This is what we use in our course.
    \end{itemize}
\end{itemize}
\end{frame}
%******************************************************************************
\begin{frame}
\frametitle{Markov decision process (MDP)}

Key elements of MDP:
\begin{itemize}
\item Sets:
\begin{itemize}
\item[-] State: the set of states $\S$
\item[-] Action: the set of actions $\A(s)$ is associated for state $s\in\S$
\item[-] Reward: the set of rewards $\mathcal{R}(s,a)$
\end{itemize}
\pause
\item Probability distribution (or called \blue{system model}):
\begin{itemize}
\item[-] State transition probability: in state $s$, taking action $a$, the probability to transit to state $s'$ is \blue{$p(s'|s,a)$}
\item[-] Reward probability: in state $s$, taking action $a$, the probability to get reward $r$ is \blue{$p(r|s,a)$}
\end{itemize}
\pause
\item Policy: in state $s$, the probability to choose action $a$ is $\pi(a|s)$
\pause
\item
\emph{Markov property}: memoryless property
\begin{align*}%\label{eq_MarkovPropertyDemo}
p(s_{t+1}|a_{t},s_t,\dots,a_0,s_0)&=p(s_{t+1}|a_{t},s_t),\nonumber\\
p(r_{t+1}|a_{t},s_t,\dots,a_0,s_0)&=p(r_{t+1}|a_{t},s_t).
\end{align*}

\end{itemize}

\pause
All the concepts introduced in this lecture can be put in the framework in MDP.
\end{frame}
%******************************************************************************
\begin{frame}
\frametitle{Markov decision process (MDP)}

The grid world could be abstracted as a more general model, \textcolor{blue}{\emph{Markov process}}.

\begin{figure}[h]
  \centering
  \includegraphics[width=0.25\linewidth]{fig_gridDemoPolicyStochastic}\qquad
  \includegraphics[width=0.3\linewidth]{fig_MPDemo}
\end{figure}

The circles represent states and the links with arrows represent the state transition.

\end{frame}
%%******************************************************************************
\begin{frame}
\frametitle{Summary}
By using grid-world examples, we demonstrated the following key concepts:
\begin{itemize}
\item State

\item Action

\item State transition, state transition probability $p(s'|s,a)$

\item Reward, reward probability $p(r|s,a)$

\item Trajectory, episode, return, discounted return

\item Markov decision process
\end{itemize}


\end{frame}

%%%%%%%%%%%%%%%%%%%%%%%%%%%%%%%%%%%%%%%%%%%%%%%%%%%%%%%%%%%%%%%%%%%%%%%%%%%%%%%%%%%%%%%%%%%%%%%%%%
%\bibliographystyle{plainnat}
%\bibliography{myOwnPub,zsyReferenceAll}
\end{document}
