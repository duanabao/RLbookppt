\documentclass[xcolor=dvipsnames,mathserif,9pt,handout]{beamer} %handout
%\usefonttheme{serif}%{structurebold}%{structuresmallcapsserif}%{serif}

\usepackage{graphicx}
\usepackage{amsmath}
\usepackage{amssymb}
\usepackage[font=footnotesize]{caption} % set the captain font size to 8 (i.e. footnotesize)
\usepackage{subfig} % uses subfloats within a single float MUST after the package {caption}!!
\usepackage{natbib}
%\usepackage{cite} % sort the reference in the article by number or alphabatic
\usepackage{color}
\usepackage{algorithm} % options: boxed [section]
\usepackage{algpseudocode} % for algorithm
%\usepackage{enumerate}
\usepackage{enumitem} % directly use itemize, easily specify indent and everything
\setlist[itemize]{leftmargin=*,label=$\bullet$}%leftmargin=*,itemsep=0pt} %topsep=5pt
\setlist[enumerate]{label={\arabic*)}}
\usepackage{hyperref}
\usepackage{wrapfig}
\usepackage{textpos}
\usepackage{bibentry} % for publication list
\makeatletter\let\saved@bibitem\@bibitem\makeatother % make hyperref and bibentry compatible!!!
\nobibliography*
\usepackage{fancybox}% shadow for image
%\usepackage{empheq} % emphasize equations
\usepackage{bm}
\usepackage{arydshln} % for dashline in table or matrix
\linespread{1.3}
\usepackage{multimedia}

\usepackage{setspace} \setstretch{1.2}

\usepackage{framed}
\colorlet{shadecolor}{black!5}
% for box, page breakable, very good!!
\usepackage[framemethod=TikZ]{mdframed}%
\mdfdefinestyle{myFrame}{%
    linecolor=gray!15!white,%gray
    outerlinewidth=0.1pt,
    roundcorner=3pt,
    skipabove=15pt, % the space before the entire box
    skipbelow=15pt, % the space after the entire box. Please see the figure 2 in the manual, very clear!
    innertopmargin=10pt,%\baselineskip,
    innerbottommargin=10pt,%\baselineskip,
    %innerrightmargin=10pt,
    %innerleftmargin=10pt,
    splittopskip=\baselineskip,
    splitbottomskip=\baselineskip,
    backgroundcolor=gray!10!white,
    frametitlerule=true,
    frametitlebackgroundcolor=gray!20!white,
    frametitleaboveskip=5pt,
    frametitlebelowskip=5pt,
}
\mdfdefinestyle{myAlgo}{%
    linecolor=gray!100!white,%gray
    outerlinewidth=0.1pt,
    roundcorner=3pt,
    skipabove=15pt, % the space before the entire box
    skipbelow=15pt, % the space after the entire box. Please see the figure 2 in the manual, very clear!
    innertopmargin=10pt,%\baselineskip,
    innerbottommargin=10pt,%\baselineskip,
    %innerrightmargin=10pt,
    %innerleftmargin=10pt,
    splittopskip=\baselineskip,
    splitbottomskip=\baselineskip,
    backgroundcolor=gray!0!white,
    frametitlerule=true,
    frametitlebackgroundcolor=gray!20!white,
    frametitleaboveskip=5pt,
    frametitlebelowskip=5pt,
}


\usepackage{tikz}
\usetikzlibrary{calc} % for calculation functions in Tikz let, in commands in Tikz
\usetikzlibrary{shapes} % for block diagram
\usetikzlibrary{chains}
\usetikzlibrary{fit}
\usetikzlibrary{arrows}
\usetikzlibrary{decorations.text} % text along path

\newcommand{\blue}[1]{\textcolor{blue}{#1}}
\definecolor{myred}{RGB}{200,0,0}
\newcommand{\red}[1]{\textcolor{myred}{#1}} %magenta purple
\newcommand{\I}{\mathcal{I}}
\newcommand{\tr}{\mathrm{tr}}
\newcommand{\Null}{\mathrm{Null}}
\newcommand{\Range}{\mathrm{Range}}
\newcommand{\one}{\mathbf{1}}
\newcommand{\rank}{\mathrm{rank}}
\newcommand{\myspan}{\mathrm{span}}
\newcommand{\mydiag}{\mathrm{diag}}
\newcommand{\D}{\mathrm{d}}
\renewcommand{\d}{\mathrm{d}}
\newcommand{\blkdiag}{\mathrm{blkdiag}}
\newcommand{\sgn}{\mathrm{sgn}}
\newcommand{\T}{\mathrm{T}}
\newcommand{\myqed}{\hfill$\blacksquare$}
\newcommand{\ep}{\varepsilon}
\newcommand{\sig}{\mathrm{sig}_a}
%\newcommand{\sigep_}[1]{\sig(\ep_{#1})}
\newcommand{\R}{\mathbb{R}}
\newcommand{\A}{\mathcal{A}}
\newcommand{\G}{\mathcal{G}}
\newcommand{\E}{\mathbb{E}}
\newcommand{\X}{\mathcal{X}}
\newcommand{\V}{\mathcal{V}}
\newcommand{\N}{\mathcal{N}}
\newcommand{\M}{\mathcal{M}}
\renewcommand{\H}{\mathcal{H}}
\renewcommand{\L}{\mathcal{B}}
\renewcommand{\S}{\mathcal{S}}
\newcommand{\xe}{x_{\text{e}}}
%\newcommand{\Null}[1]{\mathrm{Null}\left(#1\right)}
\newcommand{\sk}[1]{\left[#1\right]_\times} % skew symmetric operator
\newcommand{\dia}[1]{\mathrm{diag}\left(#1\right)} % block diagnal matrix
%\renewcommand{\span}[1]{\mathrm{span}\left\{#1\right\}} % ERROR when redefine \span
\newcommand{\Var}{\mathrm{Var}}
\newcommand{\var}{\mathrm{var}}


\graphicspath{{figures/}}

% for tikz, theorem, lemma ... environments have already been declared. You don't need to declare, or you need to use other names than theorem or lemma, such as my_theorem.
%\newtheorem{my_theorem}{Theorem}
%\newtheorem{my_lemma}{Lemma}
\newtheorem{assumption}{Assumption} % necessary for beamer
%\newtheorem{my_remark}{Remark}
\newtheorem{proposition}{Proposition} % necessary for beamer
%\newtheorem{my_corollary}{Corollary}
%\newtheorem{my_example}{Example}
%\newtheorem{my_definition}{Definition}
%\newtheorem{my_problem}{Problem}

%##################################################
\newcommand{\pagetitle}[1]{\textbf{\textcolor{BlueViolet}{$\circ$ #1}}} %!!!
\newcommand{\pagehighlight}[1]{\textbf{\textcolor{Brown}{#1}}} %!!!
\newcommand{\mypause}{\pause} % this is useful for slide show. if you don't want pause any more, just set it as blank
%\newcommand{\mybullet}{\textcolor{BlueViolet}{$\blacksquare$} }%{$\rhd$ }
%\newcommand{\myhighsign}{$\star$ }% the sign to highlight a sentence

%##################################################
% To highlight equation. Example: \begin{align*} \boxed{xxx} \end{align*}
% does not support multiline equations
% put color to \boxed math command
\newcommand*{\boxcolor}{gray}
\makeatletter
\renewcommand{\boxed}[1]{\textcolor{\boxcolor}{%
%\tikz[baseline={([yshift=-1ex]current bounding box.center)}] \node [rectangle, minimum width=1ex,rounded corners,draw] {\normalcolor\m@th$\displaystyle#1$};}}
\tikz[baseline={([yshift=-1ex]current bounding box.center)}] \node [rectangle, minimum width=2ex,rounded corners,draw] {\normalcolor\m@th$\displaystyle#1$};}}
\makeatother

%##################################################
% set my own theme
\def\structureHeight{9mm}
\usetheme[height=\structureHeight]{Rochester}
\usecolortheme[RGB={0,0,128}]{structure}
\setbeamertemplate{items}[circle]%rectangle, triangle,circle
\setbeamertemplate{blocks}[rounded][shadow=true]
\setbeamertemplate{navigation symbols}{}
%\addtobeamertemplate{frametitle}{} % specify the logo
%{
%    \begin{textblock*}{100mm}(.87\textwidth,-\structureHeight)
%        \includegraphics[height=6.6mm,width=3cm,keepaspectratio]{../common_figures_private/westlake_logo.png} % add logo
%    \end{textblock*}
%}
\addtobeamertemplate{frametitle}{\vskip4pt}{} % specify
%\setbeamerfont{frametitle}{size=\large}
\definecolor{mylightgray}{RGB}{240 240 240}
\definecolor{mykhaki}{RGB}{240 230 140}% khaki color
\definecolor{mylightYellow}{RGB}{255,255,224} % light yellow
%\setbeamercolor{beamercolor1}{bg=mylightgray, fg=black}
%\setbeamercolor{beamercolor2}{bg=mylightYellow,fg=black}%{bg=yellow!90!white, fg=black}
% background and foreground color
\setbeamercolor{background canvas}{bg=black!0!white} % background color of every slide! My previous value was black!10!white for my lecture videos!
\setbeamercolor{normal text}{bg=black!10!white} % background color for e.g. theorem environment. When canvas is 10, here it can be 20; the bg for normal text changes the color of hidden text when you use overlay

\setbeamercolor{block title}{bg=mykhaki,fg=black}
\defbeamertemplate{footline}{zsy_frameNumber}
{%
  \hspace{5pt}  \emph{Shiyu Zhao}
  \hspace*{\fill}%
  \usebeamercolor[fg]{page number in head/foot}%
  \insertframenumber\,/\,\inserttotalframenumber \vspace{0pt} \hspace{5pt}
  \vskip5pt
}
\setbeamertemplate{footline}[zsy_frameNumber]
%##################################################
\setbeamercovered{transparent=0} % a good value for transparent text is 20
% when using the overlay commands like \onslide or \uncover, the text will NOT be invisible, instead it will be like transparent
%\pause will also have the transparent effect: command \pause is easy to use: to make it invisible, change the value to zero.




\begin{document}

%%%%%%%%%%%%%%%%%%%%%%%%%%%%%%%%%%%%%%%%%%%%%%%%%%%%%%%%%%%%%%%%%%%%%%%%%%%%%%%%%
% define the author, date etc. information
%\subtitle{Mathematical and Biological Foundation for Reinforcement Learning}
\title{Lecture 2: State Value and Bellman Equation}

\author{Shiyu Zhao
\newline
\newline {\small Department of Artificial Intelligence}
\newline {\small Westlake University}
}
%\logo{\includegraphics[width=1cm,height=1cm,keepaspectratio]{NUSLogo.png}~}
%\date{\today}
\date{}
\subject{}


%%%%%%%%%%%%%%%%%%%%%%%%%%%%%%%%%%%%%%%%%%%%%%%%%%%%%%%%%%%%%%%%%%%%%%%%%%%%%%%%%

{
\setbeamertemplate{footline}{} % remove the frame number of the title page
\begin{frame}
    %\frametitle{Lecture: Networked Dynamic Systems}
    \addtocounter{framenumber}{-1} % discounter the title page, otherwise the frame number starts from 2 instead of 1
    \titlepage % this only gives the author etc. information
\end{frame}
}



\begin{frame}
\frametitle{Outline}
\begin{figure}[h]
  \centering
\includegraphics[width=0.8\linewidth]{Figure_chapterRelationship.pdf}
\end{figure}
\end{frame}
%---------------------
\begin{frame}
\frametitle{Introduction}
In this lecture, we will move
\begin{itemize}
\item from \blue{value-based methods} to \red{policy-based methods}

\item from \blue{value function methods} to \red{policy function methods} (or called policy gradient methods)
\end{itemize}
%Value function approximation:
%\begin{itemize}
%\item Goal: $\hat{v}(s,w)\rightarrow v_\pi(s)$
%\end{itemize}

\end{frame}
%---------------------
\begin{frame}
\frametitle{Outline}
\tableofcontents
\end{frame}
%--------------------------------------
\AtBeginSection[]% put it to the start of each section
{
  \begin{frame}
    \frametitle{Outline}
    \tableofcontents[currentsection]
  \end{frame}
}
\section{Basic idea of policy gradient}
%---------------------
\begin{frame}
\frametitle{Basic idea of policy gradient}

Previously, policies have been represented by tables:
\begin{itemize}
\item The action probabilities of all states are stored in a table $\pi(a|s)$. Each entry of the table is indexed by a state and an action.
\end{itemize}
{\scriptsize
\begin{table}[h!]
\centering
%\resizebox{\columnwidth}{!}{
\begin{tabular}{c|c|c|c|c|c}
  % after \\: \hline or \cline{col1-col2} \cline{col3-col4} ...
  \hline
   & \blue{$a_1$} & \blue{$a_2$} & \blue{$a_3$} & \blue{$a_4$} & \blue{$a_5$} \\
  \hline
  \blue{$s_1$} & $\pi(a_1|s_1)$ & $\pi(a_2|s_1)$ & $\pi(a_3|s_1)$ & $\pi(a_4|s_1)$ & $\pi(a_5|s_1)$ \\
  \hline
  \vdots & \vdots & \vdots & \vdots & \vdots & \vdots \\
  \hline
  \blue{$s_9$} & $\pi(a_1|s_9)$ & $\pi(a_2|s_9)$ & $\pi(a_3|s_9)$ & $\pi(a_4|s_9)$ & $\pi(a_5|s_9)$ \\
  \hline
\end{tabular}
%}
\end{table}
}

\end{frame}
%---------------------
\begin{frame}
\frametitle{Basic idea of policy gradient}

Now, policies can be represented by parameterized functions:
$$\pi(a|s,\theta)$$
where $\theta\in\R^m$ is a parameter vector.
\begin{itemize}
\pause
\item The function can be, for example, a neural network, whose input is $s$, output is the probability to take each action, and parameter is $\theta$.
\pause
\item \textbf{Advantage:} when the state space is large, the tabular representation will be of low efficiency in terms of storage and generalization.
\pause
\item The function representation is also sometimes written as $\pi(a,s,\theta)$, $\pi_\theta(a|s)$, or $\pi_\theta(a,s)$.
\end{itemize}
\end{frame}
%---------------------
\begin{frame}
\frametitle{Basic idea of policy gradient}

\textbf{Differences between tabular and function representations:
}
\pause
\begin{itemize}
\item
\blue{First, how to define optimal policies?}
\begin{itemize}
%\pause
\item[-] In the tabular case, a policy $\pi$ is optimal \red{if it can maximize \emph{every state value}}.
\pause
\item[-] In the function case, a policy $\pi$ is optimal \red{if it can maximize certain \emph{scalar metrics}}.
\end{itemize}

\end{itemize}

\end{frame}
%---------------------
\begin{frame}
\frametitle{Basic idea of policy gradient}

\textbf{Differences between tabular and function representations:
}\begin{itemize}
\item
\blue{Second, how to access the probability of an action?}
\begin{itemize}
%\pause
\item[-] In the tabular case, the probability of taking $a$ at $s$ \red{can be directly accessed} by looking up the tabular policy.
\pause
\item[-] In the function case, we \red{need to calculate the value} of $\pi(a|s,\theta)$ given the function structure and the parameter.
\end{itemize}
\end{itemize}

\end{frame}
%---------------------
\begin{frame}
\frametitle{Basic idea of policy gradient}

\textbf{Differences between tabular and function representations:
}\begin{itemize}
\item
\blue{Third, how to update policies?}
\begin{itemize}
%\pause
\item[-] In the tabular case, a policy $\pi$ \red{can be updated by directly changing the
    entries in the table}.
\pause
\item[-] In the function case, a policy $\pi$ cannot be updated in this way anymore. Instead, it \red{can only be updated by changing \emph{the parameter $\theta$}.}
\end{itemize}

\end{itemize}

\end{frame}
%---------------------
\begin{frame}
\frametitle{Basic idea of policy gradient}

The basic idea of the policy gradient is simple:
\begin{itemize}
%\pause
\item \blue{First, metrics (or objective functions) to define optimal policies:} $J(\theta)$, which can define optimal policies.
\pause
\item \blue{Second, gradient-based optimization algorithms to search for optimal policies:}
\begin{align*}
\theta_{t+1}=\theta_t + \alpha \nabla_{\theta} J(\theta_t)
\end{align*}
\end{itemize}
\pause
\vspace{10pt}
Although the idea is simple, the complication emerges when we try to answer the following questions.
\begin{itemize}
\pause
\item \blue{What appropriate metrics should be used?}
\pause
\item \blue{How to calculate the gradients of the metrics?}
\end{itemize}
\pause
These questions will be answered in detail in this lecture.
\end{frame}
%--------------------------------------
\AtBeginSection[]% put it to the start of each section
{
  \begin{frame}
    \frametitle{Outline}
    \tableofcontents[currentsection]
  \end{frame}
}
\section{Metrics to define optimal policies}
%---------------------
\subsection{Metric 1: Average value}
\begin{frame}
 \frametitle{Outline}
 \tableofcontents[currentsection]
\end{frame}
\begin{frame}
\frametitle{Metric 1: average value}
The first metric is the \blue{average state value} or simply called \blue{average value}:
\begin{align*}
\bar{v}_{\pi}
&=\sum_{s\in\S} d(s)v_{\pi}(s)
\end{align*}
\begin{itemize}
\item $\bar{v}_{\pi}$ is a weighted average of the state values.
\item $d(s)\ge0$ is the weight for state $s$.

\end{itemize}

\pause
\vspace{10pt}
Since $\sum_{s\in\S} d(s)=1$, we can interpret $d(s)$ as a \blue{probability distribution}.
Then, the metric can be written as
\begin{align*}
\bar{v}_{\pi}
&=\E_{S\sim d}[v_\pi(S)]
\end{align*}
\end{frame}
%%---------------------
%\begin{frame}
%\frametitle{Metric 1: average value}
%\textbf{Vector-product form:
%}
%\begin{align*}
%\bar{v}_{\pi}
%&=\sum_{s\in\S} d(s)v_{\pi}(s)=d^Tv_{\pi}
%\end{align*}
%\pause
%where
%\begin{align*}
%v_{\pi}&=[\dots,v_{\pi}(s),\dots]^T\in\R^{|\S|}\\
%d&=[\dots,d(s),\dots]^T\in\R^{|\S|}.
%\end{align*}
%This expression is particularly useful when we analyze its gradient.
%\end{frame}
%---------------------
\begin{frame}
\frametitle{Metric 1: average value}

\textbf{How to select the distribution $d$? There are two cases.}
\vspace{5pt}

\pause
\blue{Case 1: $d$ is \textbf{independent} of the policy $\pi$.}
\begin{itemize}
\pause
\item This case is relatively simple because the gradient of the metric is easier to calculate: $\nabla_\theta \bar{v}_\pi=d^T\nabla_\theta v_\pi$
\pause
\item In this case, we specifically denote $d$ as $d_0$ and $\bar{v}_\pi$ as $\bar{v}_{\pi}^0$.
\end{itemize}

\pause
\vspace{5pt}
How to select $d_0$?
\begin{itemize}
\item One trivial way is to treat all the states \blue{equally important} and hence select $d_0(s)=1/|\S|$.
\pause
\item Another important case is that we are only interested in \blue{a specific state} $s_0$.
For example, the episodes in some tasks always start from the same state $s_0$. Then, we only care about the long-term return starting from $s_0$.
In this case,
$$d_0(s_0)=1,\quad d_0(s\ne s_0)=0$$
In this case, \blue{$\bar{v}_\pi=v_\pi(s_0)$}
\end{itemize}
\end{frame}
%---------------------
\begin{frame}
\frametitle{Metric 1: average value}
\textbf{How to select the distribution $d$? There are two cases.}

\vspace{10pt}
\blue{Case 2: $d$ \textbf{depends} on the policy $\pi$.}

\pause
\begin{itemize}
\item A common way is to select $d$ as $d_{\pi}(s)$, which is the \blue{stationary distribution} under $\pi$. Details of stationary distribution can be found in the last lecture and the book.
%\begin{itemize}
%\pause
%\item
%One basic property of $d_\pi$ is that it satisfies
%\begin{align*}
%d_{\pi}^T P_{\pi}=d_{\pi}^T,
%\end{align*}
%where $P_{\pi}$ is the state transition probability matrix.
%\end{itemize}
\pause
\item
The interpretation of selecting $d_{\pi}$ is as follows.
\begin{itemize}
\item[-] $d_\pi$ reflects the long-run behavior of the Markov decision process under a given policy $\pi$.
\item[-] If one state is frequently visited in the long run, it is more important and deserves more weight.
\item[-] If a state is hardly visited, then we give it less weight.
\end{itemize}
\end{itemize}
\end{frame}
%---------------------
\begin{frame}
\frametitle{Metric 1: average value}

\textbf{An important equivalent expression:}

You will see the following metric often in the literature:
\begin{align*}
J(\theta)=\lim_{n\rightarrow\infty}\E\left[\sum_{t=0}^n \gamma^t R_{t+1}\right]=\E\left[\sum_{t=0}^\infty \gamma^t R_{t+1}\right].
\end{align*}

\pause
Question: What is its relationship to the metric we introduced just now?

Answer: \blue{They are the same.}
\pause
That is because
%\pause
%\begin{itemize}
%\item It starts from $S_0\sim d$ and then $A_0,R_1,S_1,A_1,R_2,S_2,\dots$
%\item $A_t\sim \pi(S_t)$ and $R_{t+1},S_{t+1}\sim p(R_{t+1}|S_t,A_t), p(S_{t+1}|S_t,A_t)$
%\end{itemize}
%Then,
\begin{align*}
J(\theta)
=\E\left[\sum_{t=0}^\infty \gamma^t R_{t+1}\right]
&=\sum_{s\in\S}d(s)\E\left[\sum_{t=0}^\infty \gamma^t R_{t+1}|S_0=s\right]\\
\visible<4->{
&=\sum_{s\in\S}d(s)v_\pi(s)\\}
\visible<5->{&=\red{\bar{v}_\pi}}
\end{align*}
\end{frame}
%---------------------
\subsection{Metric 2: Average reward}
\begin{frame}
 \frametitle{Outline}
 \tableofcontents[currentsection]
\end{frame}
\begin{frame}
\frametitle{Metric 2: average reward}
The second metric is \blue{average one-step reward} or simply \blue{average reward}:
\begin{align*}%\label{chapterGP_eq_defAverReward}
\bar{r}_{\pi}
\doteq\sum_{s\in\S} d_{\pi}(s)r_{\pi}(s)=\E[r_{\pi}(S)],
\end{align*}
where
$S\sim d_\pi$,
$$r_{\pi}(s)=\sum_{a\in\A} \pi(a|s)r(s,a)$$
$$r(s,a)=\E[R|s,a]=\sum_r rp(r|s,a)$$

\pause
\vspace{10pt}
Remarks:
\begin{itemize}
\item $\bar{r}_{\pi}$ is simply a weighted average of immediate rewards.
\item $r_{\pi}(s)$ is the average immediate reward that can be obtained from $s$.
\item $d_{\pi}$ is the stationary distribution.
\end{itemize}

\end{frame}
%%---------------------
%\begin{frame}
%\frametitle{Metric 2: average reward}
%The average reward can also be written as the inner product of two vectors.
%
%In particular, let
%\begin{align*}
%r_{\pi}&=[\dots,r_{\pi}(s),\dots]^T\in\R^{|\S|}
%\end{align*}
%be the vector of one-step immediate rewards. Then, we have
%$$\bar{r}_{\pi}= d_{\pi}^Tr_{\pi}.$$
%\end{frame}
%---------------------
\begin{frame}
\frametitle{Metric 2: average reward}

\textbf{An important equivalent expression:}

\begin{itemize}
\item Suppose an agent follows a given policy and generate a trajectory with the rewards as $(R_{1},R_{2},\dots)$.
\item The average single-step reward along this trajectory is
\begin{align*}%\label{chapterGP_eq_defAverReward2}
&\lim_{n\rightarrow \infty}\frac{1}{n}\E\Big[R_1+R_{2}+\dots+R_{n}|S_0=s_0\Big]\\
=&\lim_{n\rightarrow \infty}\frac{1}{n}\E\left[\sum_{t=0}^{n-1} R_{t+1}|S_0=s_0\right]
\end{align*}
where $s_0$ is the starting state of the trajectory.
\end{itemize}

\end{frame}
%---------------------
\begin{frame}
\frametitle{Metrics to define optimal policies - Remarks}
An important fact is that
\begin{align*}%\label{chapterGP_eq_defAverReward_equiv}
\lim_{n\rightarrow \infty}\frac{1}{n}\E\left[\sum_{t=0}^{n-1} R_{t+1}|S_0=s_0\right]&=\lim_{n\rightarrow \infty}\frac{1}{n}\E\left[\sum_{t=0}^{n-1} R_{t+1}\right]\\
\visible<2->{&=\sum_s d_{\pi}(s)r_{\pi}(s)\\}
\visible<3->{&=\red{\bar{r}_\pi}}
\end{align*}
\pause\pause\pause
Remarks:
\begin{itemize}
\item Highlight: the starting state $s_0$ does not matter.
\item The derivation of the equation is nontrivial and can be found in my book.
\end{itemize}
\end{frame}
%---------------------
\subsection{Summary of the two metrics}
\begin{frame}
 \frametitle{Outline}
 \tableofcontents[currentsection]
\end{frame}

\begin{frame}
\frametitle{Summary of the two metrics}

\begin{table}[h]
\centering
{\setstretch{2}
%\resizebox{\columnwidth}{!}{
\begin{tabular}{c|c|c|c}%{|p{0.45\textwidth}|p{0.45\textwidth}|}
  % after \\: \hline or \cline{col1-col2} \cline{col3-col4} ...
  \hline
  {Metric} & {Expression 1} & {Expression 2} & {Expression 3}\\
  \hline
  $\bar{v}_\pi $ & $\sum_{s\in\S} d(s) v_\pi(s)$ & $\E_{S\sim d}[v_\pi(S)]$ & $\lim_{n\rightarrow \infty}\E\big[\sum_{t=0}^n \gamma^t R_{t+1}\big]$ \\
  \hline
  $\bar{r}_\pi $ & $\sum_{s\in\S} d_\pi(s) r_\pi(s)$ & $\E_{S\sim d_\pi}[r_\pi(S)]$ & $\lim_{n\rightarrow \infty}\frac{1}{n}\E\big[\sum_{t=0}^{n-1} R_{t+1}\big]$ \\
  \hline
\end{tabular}
}%\setstretch
\caption{Summary of the different but equivalent expressions of $\bar{v}_\pi $ and $\bar{r}_\pi $.}
\label{chapterPG_table_summaryMetrics}
\end{table}

\end{frame}
%---------------------

\begin{frame}
\frametitle{Summary of the two metrics}

\textbf{Remark 1 about the metrics:
}
\begin{itemize}
%\pause
\item All these metrics are \blue{functions of $\pi$}.
%\pause
\item Since $\pi$ is parameterized by $\theta$, these metrics are \blue{functions of $\theta$}.
%\pause
\item In other words, different values of $\theta$ can generate different metric values.
\end{itemize}

Therefore, we can search for the \blue{optimal values of $\theta$} to maximize these metrics.
This is the basic idea of policy gradient methods.
\end{frame}
%---------------------
\begin{frame}
\frametitle{Summary of the two metrics}

\textbf{Remark 2 about the metrics:
}
\begin{itemize}
\item One complication is that the metrics can be defined in either the \blue{discounted case} where $\gamma\in(0,1)$ or the \blue{undiscounted case} where $\gamma=1$.
\item The undiscounted case is nontrivial.
\item We only consider the discounted case so far in this book. For details about the undiscounted case, see the book.
\end{itemize}

\end{frame}
%---------------------
\begin{frame}
\frametitle{Summary of the two metrics}

\textbf{Remark 3 about the metrics:
}
\begin{itemize}
\item What is the relationship between $\bar{r}_{\pi}$ and $\bar{v}_{\pi}$?
\item The two metrics are \blue{equivalent} (not equal) to each other.
Specifically, in the discounted case where $\gamma<1$, it holds that
\begin{align*}
\bar{r}_\pi=(1-\gamma)\bar{v}_\pi.
\end{align*}
Therefore, they can be \blue{maximized simultaneously}.
See the proof in the book.
\end{itemize}

\end{frame}
%--------------------------------------
\AtBeginSection[]% put it to the start of each section
{
  \begin{frame}
    \frametitle{Outline}
    \tableofcontents[currentsection]
  \end{frame}
}
\section{Gradients of the metrics}
%---------------------
\begin{frame}
\frametitle{Gradients of the metrics}

Given a metric, we next
\begin{itemize}
\item derive its gradient
\item and then, apply gradient-based methods to optimize the metric.
\end{itemize}

\pause
\vspace{10pt}
The gradient calculation is one of \blue{the most complicated parts} of policy gradient methods!
%\pause
That is because
\begin{itemize}
\item first, we need to \blue{distinguish} \red{different metrics $\bar{v}_\pi$, $\bar{r}_\pi$, $\bar{v}_\pi^0$}
%\pause
\item second, we need to \blue{distinguish} \red{discounted and undiscounted cases}.
\end{itemize}

%\pause
%\vspace{10pt}
%The calculation of the gradients:
%\begin{itemize}
%\item We will not discuss the details in this lecture.
%\item Interested readers may see my book for details.
%\end{itemize}
\end{frame}
%---------------------
\begin{frame}
\frametitle{Gradients of the metrics}
I simply give the expression of the gradient without proof:
\begin{align*}%\label{chapterGP_eq_gradientVs0}
\nabla_{\theta} J(\theta)=\sum_{s\in\S}\eta(s)\sum_{a\in\A} \nabla_{\theta}\pi(a|s,\theta)q_{\pi}(s,a)
\end{align*}
\pause
The above is a \blue{unified expression of many cases}:
\begin{itemize}
\item $J(\theta)$ can be $\bar{v}_\pi$, $\bar{r}_\pi$, or $\bar{v}_\pi^0$.
\item ``='' may denote strict equality, approximation, or proportional to.
\item $\eta$ is a distribution or weight of the states.
\end{itemize}

\pause
\vspace{10pt}
The derivation of this expression is \textbf{very complex}.

Details are not given here. Interested readers can read my book.

For most readers, it is sufficient to know this expression.
\end{frame}
%%---------------------
%\begin{frame}
%\frametitle{Gradients of the metrics}
%Some specific results:
%\begin{align*}%\label{chapterGP_eq_gradient1}
%\nabla_{\theta}\bar{r}_{\pi}\simeq\sum_s d_{\pi}(s)\sum_a \nabla_{\theta}\pi(a|s,\theta)q_{\pi}(s,a),
%\end{align*}
%
%$$\nabla_\theta\bar{v}_{\pi}=\frac{1}{1-\gamma}\nabla_\theta\bar{r}_{\pi}$$
%
%\begin{align*}%\label{chapterGP_eq_gradientVs0}
%\nabla_{\theta} \bar{v}^0_\pi=\sum_{s\in\S}\rho_\pi(s)\sum_{a\in\A} \nabla_{\theta}\pi(a|s,\theta)q_{\pi}(s,a)
%\end{align*}
%
%Details are not given here. Interested readers can read my book.
%\end{frame}
%---------------------
\begin{frame}
\frametitle{Gradients of the metrics}

\textbf{A compact and important expression of the gradient:}
\begin{align*}
\blue{\nabla_\theta J(\theta)}
&=\sum_{s\in\S}\eta(s)\sum_{a\in\A} \nabla_{\theta}\pi(a|s,\theta)q_{\pi}(s,a)\\
&\visible<2->{\blue{=\E_{S\sim \eta,A\sim \pi}\big[\nabla_{\theta} \ln \pi(A|S,\theta)q_{\pi}(S,A)\big]}}
\end{align*}

\par\noindent\rule{\textwidth}{0.1pt}

\pause\pause
\textbf{First, why is this expression useful?}
\begin{itemize}
\item Because we can use samples to approximate the gradient:
$$\nabla_{\theta}J\approx\nabla_{\theta} \ln \pi(a|s,\theta)q_{\pi}(s,a)$$
where $s,a$ are samples. This is the idea of stochastic gradient descent.
\end{itemize}

\end{frame}
%---------------------
\begin{frame}
\frametitle{Gradients of the metrics}

\textbf{A compact and important expression of the gradient:}
\begin{align*}
\blue{\nabla_\theta J(\theta)}
&=\sum_{s\in\S}\eta(s)\sum_{a\in\A} \nabla_{\theta}\pi(a|s,\theta)q_{\pi}(s,a)\\
&\blue{=\E_{S\sim \eta,A\sim \pi}\big[\nabla_{\theta} \ln \pi(A|S,\theta)q_{\pi}(S,A)\big]}
\end{align*}

\par\noindent\rule{\textwidth}{0.1pt}

\textbf{Second, how to prove the above equation?}

\pause
Proof:
Consider the function $\ln \pi$ where $\ln$ is the natural logarithm.
It is easy to see that
\begin{align*}
\nabla_{\theta} \ln \pi(a|s,\theta)= \frac{\nabla_{\theta} \pi(a|s,\theta)}{\pi(a|s,\theta)}
\end{align*}
\pause
and hence
\begin{align*}%\label{chapterGP_eq_lnpi}
\nabla_{\theta} \pi(a|s,\theta)
&=\pi(a|s,\theta)\nabla_{\theta} \ln \pi(a|s,\theta).
\end{align*}
\end{frame}
%---------------------
\begin{frame}
\frametitle{Gradients of the metrics}
\textbf{A compact and important expression of the gradient:}
\begin{align*}
\blue{\nabla_\theta J(\theta)}
&=\sum_{s\in\S}\eta(s)\sum_{a\in\A} \nabla_{\theta}\pi(a|s,\theta)q_{\pi}(s,a)\\
&\blue{=\E_{S\sim \eta,A\sim \pi}\big[\nabla_{\theta} \ln \pi(A|S,\theta)q_{\pi}(S,A)\big]}
\end{align*}

\par\noindent\rule{\textwidth}{0.1pt}

Proof (continued): Then, we have
\begin{align*}
\nabla_{\theta}J
&=\sum_s \eta(s) \sum_a \blue{\nabla_{\theta} \pi(a|s,\theta)}q_{\pi}(s,a)\\
\visible<2->{
&=\sum_s \eta(s) \sum_a \blue{\pi(a|s,\theta)\nabla_{\theta} \ln \pi(a|s,\theta)}q_{\pi}(s,a)\\}
\visible<3->{
&=\E_{\red{S\sim \eta}}\left[\sum_a \pi(a|S,\theta)\nabla_{\theta} \ln \pi(a|S,\theta)q_{\pi}(S,a)\right]\\}
\visible<4->{
&=\E_{S\sim \eta,\red{A\sim\pi}}\big[\nabla_{\theta} \ln \pi(A|S,\theta)q_{\pi}(S,A)\big]\\}
%\visible<5->{&\doteq\E\big[\nabla_{\theta} \ln \pi(A|S,\theta)q_{\pi}(S,A)\big]}
\end{align*}
\end{frame}
%---------------------
\begin{frame}
\frametitle{Gradients of the metrics}
\textbf{Remarks:} It is required by $\ln \pi(a|s,\theta)$ that for any $s,a,\theta$ $$\pi(a|s,\theta)>0$$

\pause
\begin{itemize}
\item This can be achieved by using \blue{softmax functions} that can normalize the entries in a vector from $(-\infty,+\infty)$ to $(0,1)$.
\begin{itemize}
\item[-] For example, for any vector $x=[x_1,\dots,x_n]^T$,
    $$z_i=\frac{e^{x_i}}{\sum_{j=1}^n e^{x_j}}$$
    where $z_i\in(0,1)$ and $\sum_{i=1}^n z_i=1$.
\end{itemize}
\pause
\item Specifically, the policy function has the form of
\begin{align*}
\pi(a|s,\theta)=\frac{e^{h(s,a,\theta)}}{\sum_{a'\in\A}e^{h(s,a',\theta)}}
\end{align*}
where $h(s,a,\theta)$ is another function to be learned.
\end{itemize}
\end{frame}
%---------------------
\begin{frame}
\frametitle{Gradients of the metrics}
\textbf{Remarks:}
\begin{itemize}
\item
Such a form based on the softmax function can be realized by a neural network whose input is $s$ and parameter is $\theta$. The network has $|\A|$ outputs, each of which corresponds to $\pi(a|s,\theta)$ for an action $a$. The activation function of the output layer should be softmax.
\pause
\item
Since $\pi(a|s,\theta)>0$ for all $a$, the parameterized policy is \blue{stochastic} and hence \blue{exploratory}.
\pause
\begin{itemize}
\item[-]
There also exist \blue{deterministic} policy gradient (DPG) methods. We will study in the next lecture.
\end{itemize}
\end{itemize}
\end{frame}
%--------------------------------------
\AtBeginSection[]% put it to the start of each section
{
  \begin{frame}
    \frametitle{Outline}
    \tableofcontents[currentsection]
  \end{frame}
}
\section{Gradient-ascent algorithm}
%---------------------
\begin{frame}
\frametitle{Gradient-ascent algorithm}
Now, we present \blue{the first policy gradient algorithm} to find optimal policies!
\begin{itemize}
\pause
\item[1)]
The gradient-ascent algorithm maximizing $J(\theta)$ is
\begin{align*}%\label{chapterGP_eq_REINFORCE_updateruleExp}
\theta_{t+1}
&=\theta_t+\alpha \nabla_{\theta} J(\theta_t)\nonumber\\
&=\theta_t+\alpha \E\Big[\nabla_{\theta} \ln \pi(A|S,\theta_t)q_{\pi}(S,A)\Big]
\end{align*}

\pause
\item[2)] Since the true gradient is unknown, we can replace it by a stochastic one:
\begin{align*}%\label{chapterGP_eq_REINFORCE_updaterule}
\theta_{t+1}
&=\theta_t+\alpha \nabla_{\theta} \ln \pi(a_t|s_t,\theta_t)q_\pi(s_t,a_t)
\end{align*}

\pause
\item[3)] Furthermore, since $q_\pi$ is unknown, it can be replaced by an estimate:
\begin{align*}%\label{chapterGP_eq_REINFORCE_updaterule}
\theta_{t+1}
&=\theta_t+\alpha \nabla_{\theta} \ln \pi(a_t|s_t,\theta_t)\blue{q_t(s_t,a_t)}
\end{align*}
%\pause
%There are different methods to estimate $q_\pi(s_t,a_t)$
%\begin{itemize}
%\item In this lecture, Monte-Carlo based method
%\item In the next lecture, TD method and more
%\end{itemize}
\end{itemize}
\end{frame}
%---------------------
\begin{frame}
\frametitle{Gradient-ascent algorithm}
%\begin{align*}%\label{chapterGP_eq_REINFORCE_updaterule}
%\theta_{t+1}
%&=\theta_t+\alpha \nabla_{\theta} \ln \pi(a_t|s_t,\theta_t)q_t(s_t,a_t)
%\end{align*}
%where $q_t(s_t,a_t)$ is an approximation of $q_{\pi}(s_t,a_t)$.
%\pause
\begin{itemize}
\item If $q_\pi(s_t,a_t)$ is estimated by Monte Carlo estimation, the algorithm has a specifics name, \blue{REINFORCE}.
\pause
\item REINFORCE is one of earliest and simplest policy gradient algorithms.
\pause
\item Many other policy gradient algorithms such as the actor-critic methods can be obtained by extending REINFORCE (next lecture).
\end{itemize}

\pause
\begin{figure}[t]
{\small
\begin{mdframed}[style=myAlgo,nobreak=true,frametitle={Pseudocode: Policy Gradient by Monte Carlo (REINFORCE)}]
{\fontfamily{cmss}\selectfont
\textbf{Initialization:} Initial parameter $\theta$; $\gamma\in(0,1)$; $\alpha>0$.

\textbf{Goal:} Learn an optimal policy to maximize $J(\theta)$.

\vspace{5pt}

For each episode, do

\setlength{\leftskip}{2em}
Generate an episode $\{s_0,a_0,r_1,\dots,s_{T-1},a_{T-1},r_T\}$ following $\pi(\theta)$.

\setlength{\leftskip}{2em}
For $t=0,1,\dots,T-1$:

\setlength{\leftskip}{4em} % \quad=1em
\emph{Value update:} $q_t(s_t,a_t)=\sum_{k=t+1}^T \gamma^{k-t-1} r_k$

\qquad\qquad
\emph{Policy update:}  $\theta\leftarrow \theta+\alpha \nabla_{\theta} \ln \pi(a_t|s_t,\theta)q_t(s_t,a_t)$
}%font
\end{mdframed}
}
\end{figure}
\end{frame}
%---------------------
\begin{frame}
\frametitle{Gradient-ascent algorithm}
\textbf{Remark 1: How to do sampling?}
\begin{align*}
\E_{\blue{S\sim \eta,A\sim\pi}}\Big[\nabla_{\theta} \ln \pi(A|S,\theta_t)q_{\pi}(S,A)\Big]\longrightarrow \nabla_{\theta} \ln \pi(a|s,\theta_t)q_{\pi}(s,a)
\end{align*}

\pause
%\noindent\rule{\textwidth}{0.3pt}
\begin{itemize}
\item How to sample $S$?
\begin{itemize}
\item[-] $S\sim \eta$, where the distribution $\eta$ is a long-run behavior under $\pi$.
\item[-] In practice, people usually do not care about it.
\end{itemize}
\pause
\item How to sample $A$?
\begin{itemize}
\item[-] $A\sim \pi(A|S,\theta)$. Hence, $a_t$ should be sampled following $\pi(\theta_t)$ at $s_t$.
\item[-] Therefore, \blue{policy gradient methods are on-policy}.
\end{itemize}
\end{itemize}
\end{frame}
%---------------------
\begin{frame}
\frametitle{Gradient-ascent algorithm}

\textbf{Remark 2: How to interpret this algorithm?}

\vspace{10pt}
\pause
Since
$$\nabla_{\theta} \ln \pi(a_t|s_t,\theta_t)=\frac{\nabla_{\theta}\pi(a_t|s_t,\theta_t)}{\pi(a_t|s_t,\theta_t)}$$

\pause
the algorithm can be rewritten as
\begin{align*}
\theta_{t+1}
&=\theta_t+\alpha \nabla_{\theta} \ln \pi(a_t|s_t,\theta_t)q_t(s_t,a_t)\\
&=\theta_t+\alpha \underbrace{\left(\frac{q_{t}(s_t,a_t)}{\pi(a_t|s_t,\theta_t)}\right)}_{\beta_t}\nabla_{\theta} \pi(a_t|s_t,\theta_t).
\end{align*}
\pause
Therefore, we have the important expression of the algorithm:
\begin{align*}
\blue{\theta_{t+1}=\theta_t+\alpha\red{\beta_t}\nabla_{\theta} \pi(a_t|s_t,\theta_t)}
\end{align*}
\end{frame}
%---------------------
\begin{frame}
\frametitle{Gradient-ascent algorithm}
The interpretation of
\begin{align*}
\blue{\theta_{t+1}=\theta_t+\alpha\red{\beta_t}\nabla_{\theta} \pi(a_t|s_t,\theta_t)}
\end{align*}
is as follows. Suppose that $\alpha$ is sufficiently small.

\pause
\vspace{10pt}
\textbf{Interpretation:}
\begin{itemize}
\item \blue{If $\beta_t>0$}, \red{the probability of choosing $(s_t,a_t)$ is increased:}
    \blue{\begin{align*}
    \pi(a_t|s_t,\theta_{t+1})>\pi(a_t|s_t,\theta_{t})
    \end{align*}}
\vspace{-10pt}
%The greater $\beta_t$ is, the stronger the enhancement is.
\pause
\item \blue{If $\beta_t<0$, \red{the probability of choosing $(s_t,a_t)$ is lower:}
$$\pi(a_t|s_t,\theta_{t+1})<\pi(a_t|s_t,\theta_{t})$$}
\end{itemize}
\pause
\vspace{-10pt}
\textbf{Math:} When $\theta_{t+1}-\theta_{t}$ is sufficiently small, the definition of differential implies
\begin{align*}
\pi(a_t|s_t,\blue{\theta_{t+1}})
&\approx\pi(a_t|s_t,\blue{\theta_{t}})+(\nabla_\theta \pi(a_t|s_t,\theta_t))^T(\blue{\theta_{t+1}-\theta_{t}})\\
\visible<5->{
&=\pi(a_t|s_t,\blue{\theta_{t}})+\blue{\alpha}\red{\beta_t}(\nabla_\theta \pi(a_t|s_t,\theta_t))^T(\blue{\nabla_{\theta} \pi(a_t|s_t,\theta_t)})\\
&=\pi(a_t|s_t,\blue{\theta_{t}})+\blue{\alpha}\red{\beta_t}\|\nabla_\theta \pi(a_t|s_t,\theta_t)\|^2}
\end{align*}



\end{frame}
%---------------------
\begin{frame}
\frametitle{Gradient-ascent algorithm}
\vspace{-20pt}
\begin{align*}
\theta_{t+1}
&=\theta_t+\alpha \underbrace{\left(\blue{\frac{q_{t}(s_t,a_t)}{\pi(a_t|s_t,\theta_t)}}\right)}_{\beta_t}\nabla_{\theta} \pi(a_t|s_t,\theta_t)
\end{align*}

\textbf{Interpretation (continued)}: $\beta_t$ can \blue{balance exploration and exploitation}.

\pause
\vspace{10pt}
The reason is as follows.
\begin{itemize}
\item First, \red{$\beta_t$ is \blue{proportional} to $q_{t}(s_t,a_t)$.}
$$\text{greater } q_t(s_t,a_t) \Longrightarrow \text{greater } \beta_t \Longrightarrow \text{greater } \pi(a_t|s_t,\theta_{t+1})$$
Therefore, the algorithm intends to \blue{exploit} actions with greater values.

\pause
\item Second, \red{$\beta_t$ is \blue{inversely proportional} to $\pi(a_t|s_t,\theta_t)$} (when $q_t(s_t,a_t)>0$).
$$\text{smaller } \pi(a_t|s_t,\theta_t) \Longrightarrow \text{greater } \beta_t \Longrightarrow \text{greater } \pi(a_t|s_t,\theta_{t+1})$$
Therefore, the algorithm intends to \blue{explore} actions that have low probabilities.
\end{itemize}
\end{frame}
%--------------------------------------
\AtBeginSection[]% put it to the start of each section
{
  \begin{frame}
    \frametitle{Outline}
    \tableofcontents[currentsection]
  \end{frame}
}
\section{Summary}
%---------------------
\begin{frame}
\frametitle{Summary}
Contents of this lecture:
\begin{itemize}
\item Metrics for optimality
\item Gradients of the metrics
\item Gradient-ascent algorithm
\item A special case: REINFORCE
\end{itemize}

Next lecture: Actor-critic
\end{frame}
%---------------------
%%%%%%%%%%%%%%%%%%%%%%%%%%%%%%%%%%%%%%%%%%%%%%%%%%%%%%%%%%%%%%%%%%%%%%%%%%%%%%%%%%%%%%%%%%%%%%%%%%
%\bibliographystyle{plainnat}
%\bibliography{myOwnPub,zsyReferenceAll}
\end{document}
